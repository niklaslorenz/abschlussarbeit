
\selectlanguage{ngerman}
\section*{Zusammenfassung}
In dieser Arbeit wird ein Hardware-Beschleuniger für einen rekonfigurierbaren Prozessor entwickelt, der die Berechnungszeit
von SHA-3, einer kryptographischen Hashfunktion, optimiert. Als Plattform dient dazu der i-Core, der kleine FPGA-Blöcke
zur Verfügung stellt, die frei vom Anwender konfiguriert werden können. Es wird zuerst ein kleiner Einblick in
die i-Core-Architektur als auch in SHA-3 gegeben und danach wird in einem iterativen Entwurfsverfahren der Beschleuniger konstruiert.
Dabei werden zu jedem Entwurf die getroffenen Designentscheidungen motiviert und ihre Umsetzung erklärt. In diesem Verfahren entsteht
ein Beschleuniger, der die Berechnungsgeschwindigkeit der dem SHA-3-Algorithmus zugrunde liegenden Permutation um etwa einen Faktor 36 beschleunigt.
Am Ende werden die betrachteten Entwürfe miteinander verglichen und ihre jeweiligen Stärken herausgearbeitet.

\selectlanguage{american}

~\\
~\\
\section*{Summary}
In this work, a hardware accelerator for a reconfigurable processor is developed, that optimizes the computation time of SHA-3,
a cryptographic hash function. The platform used for this purpose is the i-Core, which provides small FPGA blocks that
can be freely configured by the programmer. First, a brief overview of the i-Core architecture as well as SHA-3 is provided,
and then the accelerator is constructed through an iterative design process. In this process, the design decisions
made for each iteration are motivated and their implementation is explained. This results in an accelerator that speeds
up the computation of the permutation underlying the SHA-3 algorithm by a factor of approximately 36. Finally,
the presented designs are compared, and their respective strengths are highlighted.
