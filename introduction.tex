\section{Einleitung}

- Integrierte Mikrocontroller immer verbreiteter durch z.B. IoT
- Einsatz auch in sicherheitsrelevanten Szenarien
- Wie kann die Integrität kleiner Geräte sichergestellt werden?
- Einsatz von Attestation-Verfahren
- Grundidee: Verifiziere den Programmspeicher in periodischen Abständen
- Werkzeug: Kryptographische Hashfunktion
- Hoher Mehraufwand für den Mikrocontroller
- Spezialisierte Hardware zum Beschleunigen der Verifikation
- Rekonfigurierbare Prozessoren kostengünstig, da unabhängig vom Anwendungsfall
- Beschleuniger wird als Logik einprogrammiert
- Neben der Verifikation können die Beschleuniger auch für spezielle Anwendungsfälle benutzt werden
-> Geringere Entwicklungskosten

