\section{Finaler Entwurf}
\subsection{Entwurfsziele}
Mit den vorgestellten Überlegungen soll versucht werden das Design endlich auf die erforderliche Größe zu schrumpfen.
Die Ausführungszeit sollte dabei nicht mehr als bereits in den Überlegungen beschrieben.

\subsection{Aufbau}
[Schaubild + Erklärung]

\subsection{Funktion}
Da die Daten im BRAM nur Slice orientiert als Tiles gespeichert werden und die Eingabedaten aber als Lanes vorliegen, müssen sie erst konvertiert werden.
Diese Konvertierung findet während der vorherigen Berechnung in den B-Atomen statt.
Dabei lesen die B-Atome mehrfach den gesammten Datenblock und puffern immer einen Teil aller Lanes, bis sie sie als vollständige Tiles speichern können.
Dass dabei jede Lane mehrfach eingelesen werden muss, hat keinen Einfluss auf die Ausführunszeit des Beschleunigers, da die Berechnungszeit der modifizierten Rundenfunktion größer ist,
als die Berechnungszeit der Konvertierer und beide Systeme vollständig parallel arbeiten können.
Nachdem ein Block konvertiert wurde und die A-Atome bereit sind, werden die Tiles der Reihe nach über die Datenschnittstelle übertragen und in der Leseeinheit mit den Daten aus dem RES-Block
über ein XOR kombiniert und anschließend wieder in den RES-Block übernommen.
Für die Gamma-Berechnung werden die Slices über den Lesebus aus dem RES-Block gelesen an die Recheneinheit sowie an die Kommunikationseiheit übergeben.
Die Berechnung selbst funktioniert genau wie im zweiten Entwurf, tatsächlich wird die Recheneinheit aus dem vorherigen Entwurf wiederverwendet.
Die Ergebnisse werden anschließend von der Kommunikationseinheit und der Recheneinheit über den Schreib-Bus in den GAM-Block geschrieben.
Anschließend wird die Links-Rotation mit Hilfe des Rho-Puffers durchgeführt. Die Daten werden aus dem GAM-Block über den Datenbus an den Puffer übertragen.
Zuerst wird der Puffer mit den Initialdaten gefüllt, dann werden wie im Abschnitt [welcher Abschnitt ist denn das?] die Tiles nach und nach in den Puffer eingeschoben
und die Erbebnis-Tiles werden über den Schreib-Bus wieder in den GAM-Block übernommen.
Die Rechts-Rotation wird genau so durchgeführt, nur werden dieses Mal die Ergebnisse in den RES-Block geschrieben anstatt in den GAM-Block.
Hier wird auch ein Vorteil des Bus-Designs deutlich: Da beide Blöcke ihre Daten über den Bus empfangen, braucht für die Rechts-Rotation nur das
Write-Enable-Signal verändert werden und die Ergebnisse müssen nicht auf eine andere Eingabe umgeleitet werden.
Die Rho und Gamma Funktionen werden so oft berechnet, bis der Block gemäß der KECCAK-P Funktion verarbeitet wurde.
Danach kann entweder über das Kontrollsignal bestimmt werden, ob der nächste Block eingelesen werden soll, oder das Ergenis ausgegeben werden soll.
Da das Ergebnis aus den ersten 4 Lanes der gespeicherten Daten besteht, müssen die Tiles wieder zu Lanes konvertiert werden.
Dabei ist keine Kommunikation zwischen den Atomen notwendig, da der Datenblock Spalten-orthogonal aufgeteilt wurde und so alle benötigten Lanes sich in Atom 0 befinden.
Es werden über den Datenbus alle 64 Tiles nacheinander ausgelesen und die jeweils 4 ersten Bits in einem extra Puffer gespeichert.
Liegt das gesammte Ergebnis im Puffer, so wird es über das Interface nach außen ausgegeben.

\subsection{Bewertung}

\subsection{Weitere Optimierungsansätze}

\subsubsection{Reduktion auf 1 Atom}
Die Berechnung wurde ursprünglich auf zwei Atome aufgeteilt, damit die Datenmenge reduziert werden kann, die ein Atom halten muss.
Durch die Einführung des BRAM fällt diese Optimierung weg, da der BRAM mehr als genug Platz bereitstellt.

\subsubsection{Erweiterung des Speicherinterfaces auf mehr Tiles}
Die Aufteilung des Speichers in zum Beispiel 4 Tiles erlaubt es eine größere Menge an Tiles gleichzeitig für die Berechnung von Rho bereit zu stellen.
Da in einem zwei Atom Ansatz die Gamma-Funktion durch die Bandbreite des Datenkanals begrenzt ist, würde dieser Teil keine Beschleunigung erfahren.
In einem Ein-Atom-Ansatz hingegen würden allerdings auch hier noch mehr Slices gleichzeitig berechnet werden können.

\subsubsection{Erweiterung des Rho-Puffers}
Durch 1 Atom Ansatz müssen alle Lanes in einem Atom berechnet werden, kann der Puffer erweitert werden oder braucht das zu viel Platz?
Müssen mehr als 2 Rotationen durchgeführt werden?

\subsubsection{Auslagerung der Rho-Berechnung}
Auch wenn alle Daten in einem Atom gespeichert werden, könnte es sinnvoll sein, die Berechnung von Rho auf verschiedene Atome zu verteilen.
[Schaubild]. So können beim Filtern der Lanes die restlichen Lanes an ein anderes Atom übertragen werden, das die Rotation durchführt.
So kann die Größenbeschränkung des Puffers umgangen werden. Wird auch dieser Prozess durch eine Pipeline realisiert, beträgt die maximale Bandbreite hierfür
max k s.t. k * 7 <= 32
=> k = 4 Slices. Wird der Puffer im Hauptatom noch ein wenig erweitert, sodass maximal 6 Lanes gleichzeitig ausgelagert werden müssen, so steigt die Bandbreite sogar
auf 5 Slices. Da allerdings immer zwei Ports für die Zusammensetzung der Daten verantwortlich sind, müssten immer 6 Slices gleichzeitig gelesen werden und die
Restlichen in einem weiteren Puffer zwischengehalten werden was die Komplexität weiter erhöht. Außerdem ist ein solcher Ansatz nicht mehr symmetisch,
heißt es kann nicht der gleiche Beschleuniger auf beide Atome geladen werden.
