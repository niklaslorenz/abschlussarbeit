Die drei vorgestellten Designs für den Beschleuniger bauen aufeinander auf mit dem Ziel immer mehr Leistung für weniger Platzbedarf einzutauschen.
In diesem Kapitel werden alle Designs miteinander verglichen was sowohl ihre Größe angeht, als auch ihr Leistungspotential.
Das konkrete Ziel dieser Arbeit bestand darin einen Beschleuniger zu entwerfen, der speziell die Anforderungen der i-Core-Architektur erfüllt.
Daher wird in diesem Kapitel hauptsächlich der i-Core als Vergleich herangezogen für die Ausführung von Software-Implementierungen.
Außerdem werden wir uns bei der Untersuchung der Ausführungszeiten auf die KECCAK-p-Funktion beschränken,
da diese den Hauptaufwand darstellt und gerade bei größeren Datenmengen Schritte wie die Initialisierung und das Auslesen des Hashs am Ende vernachlässigbar sind.
Bei Vergleichen zwischen Beschleunigern mit Software-Berechnungen dient immer die Implementierung aus Anhang \ref{cha:anhang1} als Bezug.

\section{Syntheseergebnisse}
\begin{table}
    \centering
    \begin{tabular}{lrrrrr}
    & A-Atome & B-Atome & LUTs & FFs & BRAM Bänke \\
    \hline
    1. Entwurf & 1 & 0 & 3314 & 1845 & 0 \\
    2. Entwurf & 2 & 0 & 4643 & 1865 & 0 \\
    3. Entwurf & 2 & 2 & 1205 & 628 & 2
    \end{tabular}
    \label{tab:synth_ergebniss}
    \caption{Interessante Größen der synthetisierten Designs}
    \small
    Sämtliche Größen beziehen sich auf die A-Atome
\end{table}
In Tabelle \ref{tab:synth_ergebniss} sind die Größen der verschiedenen Beschleunigeriterationen noch einmal zusammengefasst.
Während die ersten beiden Iterationen alle Funktionalitäten in die A-Atomen integrieren, die die tatsächliche Berechnung der KECCAK-p-Funktion durchführen,
benötigt die dritte Iteration noch zwei zusätzliche B-Atome, in denen die Eingabedaten so umgeordnet werden, dass sie in den BRAM der A-Atome geschrieben werden können.
Da in dem dritten Entwurf die Obergrenze von 1600 LUTs für die A-Atome noch nicht erreicht ist, kann diese Funktionalität auch in die Beschleuniger integriert werden.
Je weniger Atome für den Beschleuniger benötigt werden, desto schneller kann der Beschleuniger geladen werden.
Performance-technisch ist es jedoch sinnvoller diese Funktionen zu trennen. Da die A-Atome bei der Berechnung sowohl den BRAM als auch ihre Kommunikationsschnittstelle
voll auslasten, kann die Aufgabe der B-Atome nicht gleichzeitig in den A-Atomen durchgeführt werden. Daher müsste für die Zusammenführung der Atome die Einlese-Phase deutlich verlängert werden.
Für das Verarbeiten großer Datenmengen ist es daher sinnvoll die leicht erhöhte Konfigurationszeit des Beschleunigers in Kauf zu nehmen.

\section{Ausführungszeit}
Zur Bestimmung der Ausführungszeit des Beschleunigers verwenden wir hier die Simulation. Das hat den Vorteil,
dass auch die vorherigen Entwürfe mit verglichen werden können, die zu groß für den i-Core sind.
Da alle Entwürfe statisch sind in ihrer Ausführungszeit, die Abarbeitung eines Blocks also immer die gleiche Zeit benötigt,
liefert diese Herangehensweise genauere Werte als die Zeitmessung es auf dem i-Core tun würde.
Da aller Entwürfe den Betrieb auf dem 50MHz Takt des i-Cores zulassen, können die Taktzahlen direkt miteinander verglichen werden, siehe Tabelle \ref{tab:zeiten_alle_iterationen}.
\begin{table}
    \centering
    \begin{tabular}{lrrr}
        & Berechnungsdauer & Einlesedauer & Gesamtdauer\\
        \hline
        1. Entwurf & 24 Takte & 34 Takte & 58 Takte \\
        2. Entwurf & 509 Takte & 34 Takte & 543 Takte \\
        3. Entwurf & 1968 Takte & 36 Takte & 2004 Takte
    \end{tabular}
    \label{tab:zeiten_alle_iterationen}
    \caption{Ausführungszeiten der verschiedenen Beschleuniger-Entwürfe}
    \small Alle Zeitangaben beziehen sich auf die Abarbeitung eines Datenblocks. Initialisierung und Auslesen des Hashs sind nicht einberechnet.
\end{table}
Die Einlesedauer der Iterationen unterscheidet sich dabei fast gar nicht. In den ersten beiden Entwürfen liegt jede der 17 Lanes aus den Eingabedaten
nacheinander jeweils einen Kontrollschritt, also zwei Takte, an beiden Atomen an und die Atome wählen jeweils die Lanes aus, die sie benötigen
(im ersten Entwurf liest das einzige Atom alle Lanes ein).
Im dritten Entwurf hingegen lesen beide A-Atome gleichzeitig verschiedene Slices, die ihnen von den B-Atomen bereitgestellt werden.
Da in jedem Kontrollschritt jedoch nur jeweils vier Tiles zu je 13 Bits übertragen werden anstatt einer ganzen Lane, werden 16 statt der vorherigen 13 Kontrollschritte benötigt.
Diese Effekte heben sich gegenseitig auf. Eine weitere Optimierung der Einlesezeiten ist auch nicht weiter nötig. Im ersten Entwurf ist sie nicht möglich,
da die Kommunikationsschnittstelle vollständig ausgelastet ist und in den beiden letzten Entwürfen beträgt das Einlesen einen zu kleinen Teil an der gesamten Berechnung.

Die tatsächliche Berechnung hingegen nimmt mit jedem Entwurf deutlich mehr Zeit in Anspruch. Das ist alleine der Sequenzialisierung geschuldet.
Statt wie vorher die gesamte Rundenfunktion in einem Takt abzuarbeiten, werden im zweiten Entwurf Rho in einem Schritt und Gamma/Theta in 16 Schritten für jeweils 2 Slices pro Atom berechnet.
Hinzu kommen noch ein paar Takte durch die Verzögerung zwischen den Atomen, sodass die Berechnung der erweiterten Rundenfunktion im zweiten Entwurf insgesamt 21 Takte benötigt.
Der dritte Entwurf verwendet die gleiche Vorgehensweise für die Berechnung von Gamma/Theta, braucht allerdings drei Takte länger durch die Verzögerung des BRAM
und berechnet außerdem auch Rho sequenziell in 57 Takten. In Tabelle \ref{tab:zeiten_iteration_3} sind die Ausführungszeiten
für die verschiedenen Teilfunktionen im dritten Entwurf noch einmal zusammengefasst.
\begin{table}
    \centering
    \begin{tabular}{lrrr}
    Funktion & Zeit (in Takten) & Iterationen in KECCAK-p & Anteil an KECCAK-p \\
    \hline
    Initialisierung & 18 & - & - \\
    Einlesen & 36 & 1 & 1,8\% \\
    Theta & 24 & 1 & 1,2\% \\
    Gamma & 24 & 24 & 28,74\% \\
    Rho & 57 & 24 & 68,26\% \\
    KECCAK-p & 2004 & 1 & 100\% \\
    Auslesen & 27 & - & -
    \end{tabular}
    \label{tab:zeiten_iteration_3}
    \caption{Ausführungszeiten des finalen Beschleunigers}
\end{table}
Die KECCAK-p-Funktion, für die wir uns eigentlich bei der Berechnung von SHA-3 interessieren, setzt sich zusammen aus dem Einlesen des Datenblocks,
einer Berechnung von Theta, sowie 24 Berechnungen von Rho und Gamma.

\section{Weitere Optimierungsansätze}
Der dritte Entwurf des Beschleunigers erfüllt zwar alle geforderten Voraussetzungen, trotzdem lassen sich noch weitere Verbesserungen vornehmen, um die Leistungsfähigkeit zu erhöhen.
Daher wollen wir uns hier noch ein paar dieser möglichen Optimierungen anschauen.

\subsection{Auslagerung der Ergebniskonvertierung}
Das Ergebnismodul nimmt unnötig viel Platz im Atom ein und ist eigentlich nur deshalb in den A-Atomen enthalten, weil der Platz nicht weiter benötigt wird.
Es lässt sich aber auch in die B-Atome verschieben, die auch die Konvertierung für die Eingabe übernehmen.
So kann noch ein wenig mehr Platz für andere Optimierungen geschaffen werden, die die Berechnungsgeschwindigkeit weiter verbessern.

\subsection{Reduktion auf ein A-Atom}
Die Berechnung wurde ursprünglich auf zwei Atome aufgeteilt, damit die Datenmenge reduziert werden kann, die ein Atom halten muss.
Durch die Verwendung des BRAM für den Datenspeicher fällt dieser Aufwand weg, da der BRAM mehr als genug Platz bereitstellt.
Ein Beschleuniger, der nur ein A-Atom verwendet, bietet vor allem den Vorteil, dass für die Konvertierung der Eingabedaten auch nur ein B-Atom benötigt wird.
Der gesamte Beschleuniger benötigt also nur zwei der fünf Atome. Abhängig vom Anwendungsfall können so auch mehrere kleine Beschleuniger nebeneinander existieren,
ohne dass die Atom-Container zwischendurch rekonfiguriert werden müssen.
Außerdem ist die Berechnung selbst nicht mehr durch die Kommunikationsschnittstelle zwischen den Atomen beschränkt, wodurch weitere Optimierungen möglich werden.
Die Reduktion auf ein A-Atom alleine bringt jedoch keine direkte Leistungsverbesserung. Im Gegenteil, da aktuell beide A-Atome parallel sowohl Rho als auch Gamma berechnen,
würde die Reduktion alleine die Berechnungszeit etwa verdoppeln. Auch das Einlesen der Datenblöcke dauert länger, da der gesamte Block in ein einziges Atom übertragen werden muss,
was wiederum durch die Kommunikationsschnittstelle beschränkt ist. 

\subsection{Erweiterung der BRAM-Schnittstelle}
Aktuell können an einem Port des BRAM immer jeweils zwei Tiles gleichzeitig gelesen und geschrieben werden.
Die Erweiterung der Speicherschnittstelle auf zum Beispiel 4 Tiles erlaubt es,
eine größere Menge an Tiles gleichzeitig für die Berechnung von Rho bereit zu stellen.
Die Rho-Funktion könnte somit doppelt so schnell berechnet werden. Da die Berechnung von Rho mit 57 Takten
etwa 70\% der 81 Takte für die Berechnung der erweiterten Rundenfunktion beansprucht (siehe \ref{tab:zeiten_iteration_3}),
ist nochmal mit einer weiteren Beschleunigung von $S_{WideBRAM} = \frac{81\ Takte}{81\ Takte - (57\ Takte / 2)} \thickapprox 1,54$ zu rechnen.
Die restlichen 30\% der Berechnungszeit für die erweiterte Rundenfunktion werden für die Gamma-Funktion verwendet.
Diese kann durch die Erweiterung der Speicherschnittstelle nicht weiter beschleunigt werden,
da sie die Kommunikationsschnittstelle zwischen den Atomen bereits den limitierende Faktor darstellt.
Unklar ist noch, ob der vorhandene Platz für diese Optimierung ausreicht.

\subsection{Erweiterung des Rho-Puffers}
Möchte man wie oben beschrieben den Beschleuniger auf ein Atom reduzieren, kann auch der Rho-Puffer,
in dem die Bit-Rotationen blockweise auf mehreren Lanes gleichzeitig durchgeführt werden, erweitert werden.
Im aktuellen Entwurf ist das nicht mehr möglich, da genau eine Links- und eine Rechtsrotation auf
etwa jeweils der Hälfte der in den Atomen gespeicherten Lanes durchgeführt werden.
Links- und Rechtsrotationen lassen sich nicht zusammenlegen, was damit zusammenhängt in welche
Richtung die Lanes aus dem Speicher gelesen und geschrieben werden.
Findet jedoch die gesamte Berechnung nur in einem Atom statt, sind für die vollständige Abarbeitung
zwei Links- und zwei Rechts-Rotationen notwendig. Erweitert man den Puffer, sodass jeweils die beiden Links- und die beiden Rechts-Rotationen
zusammengelegt werden können, kann dadurch verhindert werden, dass die Berechnung der Rho-Funktion länger dauert.

\subsection{Auslagerung der Rho-Berechnung}
Es ist zu erwarten, dass die Erweiterung des Rho-Puffers viel zusätzlichen Platz in Anspruch nehmen wird.
Sollte das Atom dadurch die maximale Größe überschreiten, kann eventuell ein Teil der Berechnung von Rho in ein anderes Atom wieder ausgelagert werden.
Aktuell werden in jedem Takt jeweils 4 Bits aus maximal 7 Lanes, also maximal 28 Bit in den Puffer geschrieben und gleichzeitig gelesen.
Dieser Datenverkehr ist über die Kommunikationsschnittstelle durchaus gleichzeitig in beide Richtungen möglich, er kann sogar noch verdoppelt werden,
wenn man auch die BRAM-Schnittstelle wie oben beschrieben erweitert. Jenachdem wie groß diese Konstruktion wird, kann sie auch in das B-Atom integriert werden,
sodass der Beschleuniger trotzdem nur aus zwei Atomen besteht. Das B-Atom würde dann während der Berechnung von Rho mit dem A-Atom zusammenarbeiten und
zwischendurch den nächsten Datenblock einlesen.

\subsection{Erhöhung der Berechnungsfrequenz}
Der Beschleuniger selbst erlaubt eine Taktfrequenz von 200MHz, was in diesem Fall das Maximum für das verwendete FPGA ist. Der i-Core selbst
läuft jedoch nur mit einer Frequenz von 50MHz. Da der Beschleuniger seinen Takt mit dem i-Core teilt, damit die Synchronisierung am einfachsten ist,
läuft er jedoch sehr viel langsamer, als er eigentlich könnte. Verwendet man einen Takt von 200MHz
für die Berechnung der Rho-Funktion, die vollständig in den Atom erfolgt und keinerlei Kommunikation benötigt, erhält man eine Beschleunigung von
$S_{200MHz} = \frac{81\ Takte / 50MHz}{(81\ Takte - 57\ Takte) / 50MHz + 57\ Takte / 200MHz} \thickapprox 2,12$.
Für die Herkunft der genauen Taktzahlen siehe Tabelle \ref{tab:zeiten_iteration_3}. Auch hier kann wieder nur die Rho-Funktion beschleunigt werden,
da die Kommunikation zwischen den Atomen durch den i-core wieder durch den 50MHz-Takt beschränkt ist.

\section{Gemessene Beschleunigung}
Wie bereits erwähnt, sind die entworfenen Beschleuniger strikt deterministisch in der Hinsicht, dass die Dauer der Berechnung immer dieselbe Zeit benötigt.
Daher kann aus den vom Beschleuniger benötigten Takten sowie der Taktfrequenz direkt die Ausführungszeit des Beschleunigers bestimmt werden.
Bei der Software-Berechnung ist das nicht so einfach möglich, da zum Beispiel Speicherzugriffe aufgrund der Speicherstruktur unterschiedliche Ausführungszeiten benötigen.
So kann die Anzahl der auftretenden Cache Misses einen großen Einfluss auf die Berechnungsdauer haben. Auch ist nicht klar, ob wirklich jede Operation des Befehlssatzes
in genau einem Takt ausgeführt werden kann. Durch Datenabhängigkeiten kann es notwendig sein, dass die weitere Abarbeitung auf Ergebnisse vorheriger Instruktionen,
die sich noch in der Pipeline befinden, warten muss. Die beste Möglichkeit zur Bestimmung der Laufzeit der Software-Berechnung ist daher die tatsächliche Zeitmessung.
Diese Messung ist natürlich etwas ungenau, weshalb in 5 Durchläufen jeweils 10.000 Iterationen der KECCAK-p-Funktion durchgeführt werden, woraus dann die mittlere Berechnungsdauer bestimmt wird.
Die Ergebnisse dieser Messungen sowie der berechnete Durchschnitt $\tilde{t}$ sind in Tabelle \ref{tab:software_zeitmessung} aufgeführt.
\begin{table}
    \centering
    \begin{tabular}{lrrrrrr}
        Messung & 1 & 2 & 3 & 4 & 5 & Durchschnitt ($\tilde{t}$)\\
        \hline
        Zeit (s) & 14,505455 & 14,505454 & 14,505463 & 14,505464 & 14,505452 & 14,5054576
    \end{tabular}
    \label{tab:software_zeitmessung}
    \caption{Ausführungszeit der Software-Berechnung}
\end{table}
Aus dieser Messung ergibt sich für eine einzige Berechnung der KECCAK-p-Funktion eine Zeit von
\begin{align*}
    t_{SW} & = \tilde{t}/10000 = 14,5054576 s / 10000 = 1450,54576 \mu s
\end{align*}
Die Ausführungszeit der KECCAK-p-Funktion auf dem finalen Beschleuniger und der daraus resultierende Speedup kann
mit Hilfe der Tabelle \ref{tab:zeiten_iteration_3} aus dem 50MHz-Takt des i-Core wie folgt exakt berechnet werden:
\begin{align*}
    t_{ACC} & = \frac{2004\ Takte}{50 * 10^6Hz} = 0,00004008s = 40,08 \mu s \\
    S_{ACC} & = \frac{t_{SW}}{t_{ACC}}\frac{1450,54576 \mu s}{40,08 \mu s} \thickapprox \mathbf{36,2}
\end{align*}

\section{Theoretische Beschleunigung}
Neben der tatsächlichen Ausführungszeit auf dem i-Core wollen wir uns noch eine andere Metrik anschauen, um den Beschleuniger zu bewerten.
Dazu betrachten wir die Anzahl an elementaren Operationen, die die Software auf einem virtuellen System benötigt, um die KECCAK-p-Funktion zu berechnen.
Als elementare Operationen zählen dabei Instruktionen, von denen man erwarten kann, dass sie von jedem modernen Prozessor in jeweils einem Takt berechnet werden können.
Als solche Instruktionen zählen in diesem Fall XODER, UND, ODER, NEGATION (NEG), das Kopieren, eine Rotation um ein Bit nach links (ROL) sowie ein Bitshift variabler Länge.
Weiter unterschieden werden muss allerdings zwischen 32-Bit- und 64-Bit-Operationen. In der Tabelle \ref{tab:software_instruktionen} ist
die Anzahl der jeweils benötigten 64-Bit-Operationen aufgeführt, die für die einzelnen Teilfunktionen von KECCAK-p benötigt werden, die sich aus insgesamt 24 Runden zusammensetzt.
Die Anzahl an benötigten 32-Bit-Operationen ist dann doppelt so groß. Wie auch in der Implementierung in Anhang \ref{cha:anhang1} beschrieben, ist eine 64-Bit-Rotation,
wie sie zum Beispiel von $\rho$ verwendet wird, dabei aus zwei Shifts und einem ODER zusammengesetzt.
Außerdem werden sämtliche Operationen, die nicht direkt zur Ergebnisberechnung dienen,
wie zum Beispiel Zählvariablen oder Indexberechnungen für Felder, ignoriert.
\begin{table}
    \centering
    \begin{tabular}{lrrrrrrrr}
        & XODER & UND & ODER & NEG & Kopie & ROL & Shift & Gesamt \\
        \hline
        Theta & 50 & 0 & 0 & 0 & 0 & 5 & 0 & 55 \\
        Rho-Pi & 0 & 0 & 24 & 0 & 49 & 0 & 48 & 121 \\
        Chi & 25 & 25 & 0 & 25 & 25 & 0 & 0 & 100 \\
        Iota & 1 & 0 & 0 & 0 & 0 & 0 & 0 & 1 \\
        Rnd & 76 & 25 & 24 & 25 & 74 & 5 & 48 & 277\\
        KECCAK-p & 1824 & 600 & 576 & 600 & 1776 & 120 & 1152 & 6648
    \end{tabular}
    \label{tab:software_instruktionen}
    \caption{Instruktionen der Software-Funktionen}
\end{table}
Da wir gefordert haben, dass jede dieser Operationen in jeweils einem Takt berechnet werden soll,
ergibt sich aus der Anzahl der Operationen auch gleich die Anzahl der benötigten Takte.
Auf diese Weise können wir wieder einen Speedup für den Beschleuniger berechnen, indem wir die benötigten Operationen
der Software mit den benötigten Takten des Beschleunigers vergleichen. Hier können auch alle Entwürfe einbezogen werden,
da wir für unsere virtuelle Umgebung keine Begrenzungen für die Größe der Beschleuniger vorgegeben haben und somit der Vergleich auch sinnvoll ist.
In Tabelle \ref{tab:theoretischer_speedup} ist sowohl der theoretische Speedup sowohl bezüglich 32-Bit- als auch 64-Bit-Operationen aufgeführt.
\begin{table}
    \centering
    \begin{tabular}{lrr}
        & 32-Bit-Speedup & 64-Bit-Speedup \\
        \hline
        1. Entwurf & 229,24 & 114,62 \\
        2. Entwurf & 24,49 & 12,25 \\
        3. Entwurf & \textbf{6,63} & 3,32
    \end{tabular}
    \label{tab:theoretischer_speedup}
    \caption{Theoretischer Speedup der verschiedenen Entwürfe}
\end{table}
Dieser theoretische Speedup beschreibt den Faktor, wie viel der tatsächlichen Berechnung der Beschleuniger in jedem Schritt mehr erledigt,
als die Software es im Optimalfall tut. Vergleicht man den theoretischen Speedup mit dem gemessenen Speedup, so stellt man fest,
dass der gemessene Speedup des finalen Entwurfs mit 36,2 um einen Faktor 5,46 größer ist als der theoretische Speedup von 6,63
(wir verwenden hier den 32-Bit-Speedup, da der i-Core eine 32-Bit-Plattform ist). Dieser Faktor beschreibt die gewonnene Geschwindigkeit,
die dadurch entsteht, dass der Beschleuniger keine Kontrollstrukturen wie Schleifen, Speicherzugriffszeiten, oder mehrere Takte für eine Instruktion benötigt.
