\section{Begriffe}
\begin{tabularx}{\linewidth}{@{}>{\bfseries}l@{\hspace{.5em}}X@{}}
	Atom: & Beschleuniger, oder Teil eines Beschleunigers, der in einen Atom-Container geladen werden kann \\
	Atom-Container: & Programmierbarer Logikblock in einem rekonfigurierbaren Prozessor \\
	Atom-Index: & Laufzeitparameter des Beschleunigers, der die genaue Aufgabe des Atoms bestimmt \\
	Lane: & Ein eindimensionaler 64-Bit langer Ausschnitt aus einem State Array \\
	Slice: & Ein ein Bit breiter vertikaler Ausschnitt aus allen Lanes, insgesamt 25 Bits groß \\
	State Array: & Die dreidimensionale Darstellung des internen Zustandsvektors der KECCAK-p-Funktion \\
	Schwammkonstruktion: & Teil des SHA-3-Algorithmus, siehe \ref{cha:sha3} \\
	Tile: & Die 13 oberen oder 13 unteren Bits eines Slices (es gibt ein Bit, das in beiden enthalten ist)
\end{tabularx}
\newpage
\section{Abkürzungen}
\begin{tabularx}{\linewidth}{@{}>{\bfseries}l@{\hspace{.5em}}X@{}}
	AC: & Atom Conainer \\
	AGU: & Address Generation Unit \\
	BPP: & Bounded Error Probabilistic Polynomial Time \\
	BQP: & Bounded Error Quantum Polynomial Time \\
	BRAM: & Block RAM \\
	CLB: & Configurable Logic Block \\
	DSP: & Digital Signal Processor \\
	FPGA: & Field Programmable Gate Array \\
	IoT: & Internet of Things \\
	LSU: & Load-Store-Unit \\
	LUT: & Lookup Table \\
	SHA: & Secure Hash Algorithm \\
	SI: & Spezialinstruktion \\
	TLM: & Tile Local Memory \\
	VLCW: & Very Long Control Word
\end{tabularx}