\section{Vorgehensweise}
Ziel ist es, einen Beschleuniger zu entwickeln, der die Berechnung von SHA-3 möglichst effizient durchführt.
Dabei müssen aber alle Voraussetzungen der gewählten Architektur eingehalten werden. Im Falle des icore sind
das die Größenbeschränkung von maximal fünf Atomen mit jeweils 1600 LUTs, sowie das Implementieren der
Kommunikationsschnittstelle der Atome, sodass sie sowohl untereinander, als auch mit dem icore, Daten austauschen können.
Da quasi der gesamte Rechenaufwand von SHA-3 aus der wiederholten Berechnung der KECCAK-p-Funktion besteht,
soll der hier entworfene Beschleuniger genau diese KECCAK-p-Funktion berechnen.
Für den Entwurf verwenden wir hier ein iteratives Design-Verfahren. Angefangen wird mit einem Beschleuniger,
der die Architekturbeschränkungen komplett ignorieren darf, um eine maximale Berechnungsgeschwindigkeit zu erzielen.
Der Platz des Beschleunigers sollte dennoch sinnvoll genutzt werden. Auf diesem Entwurf aufbauend können dann
Strategien entwickelt werden, um die Berechnung so aufzuteilen, dass der entstehende Beschleuniger nach und nach die
Voraussetzungen der Architektur erfüllt.