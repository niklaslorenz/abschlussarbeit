\section{Vorgehensweise}
Ziel ist es einen Beschleuniger zu entwickeln, der die Berechnung von SHA-3 möglichst effizient durchführt.
Da alle SHA-3-Funktionen die gleiche Instanz von KECCAK-p verwenden, ist die Wahl der konkret zu implementierenden
SHA-3-Funktion praktisch egal. Sie unterscheiden sich nur in der Größe der Eingabeblöcke, die mit dem internen
State Array kombiniert werden, sowie der Ausgabelänge. Daher legen wir uns hier auf SHA3-256 fest, die der Beschleuniger implementieren soll.
Dabei müssen aber alle Voraussetzungen der gewählten Architektur eingehalten werden. Im Falle des i-Core sind
das die Größenbeschränkung von maximal fünf Atomen mit jeweils 1600 LUTs, sowie das Implementieren der
Kommunikationsschnittstelle der Atom-Container, sodass sie sowohl untereinander, als auch mit dem i-Core Daten austauschen können.
Da quasi der gesamte Rechenaufwand von SHA-3 aus der wiederholten Berechnung der KECCAK-p-Funktion besteht,
soll der hier entworfene Beschleuniger genau diese KECCAK-p-Funktion berechnen.
Der Beschleuniger soll in einem iterativen Verfahren entworfen werden, wobei die Anforderungen am Anfang noch
nicht eingehalten werden müssen. Angefangen wird mit einem Beschleuniger,
der die Architekturbeschränkungen komplett ignoriert, um eine maximale Berechnungsgeschwindigkeit zu erzielen.
Der Platz des Beschleunigers sollte dennoch sinnvoll genutzt werden. Auf diesem Entwurf aufbauend können dann
Strategien entwickelt werden, um die Berechnung so aufzuteilen, dass der entstehende Beschleuniger nach und nach die
Voraussetzungen der Architektur erfüllt.