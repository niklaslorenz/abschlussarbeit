\section{Erste Iteration}
\subsection{Entwurfsziele}

Die Architektur gibt eine maximale Größe von 1600 LUTs für seine Beschleunigerblöcke vor und ein Beschleuniger kann aus maximal 5 solcher Blöcke bestehen.
Weiterhin permutiert die Keccak-Funktion ein 1600 Bit breites Feld, wobei jedes Bit von vielen anderen Bits an verschiedenen Stellen im Feld abhängt.
Da die verschiedenen Blöcke nicht beliebig miteinander kommunizieren können, sondern nur über eine taktsynchrone Schnittstelle mit sehr begrenzter Bandbreite,
folgt aus den beiden Beobachtungen die Vermutung, dass ein guter Beschleuniger aus so wenig Blöcken wie möglich besteht.
Ziel des ersten Entwurfs war daher einen Beschleuniger zu entwickeln, der die Keccak-Permutation so schnel wie möglich in einem Block berechnet.
Dabei war es erlaubt die Größenbeschränkung von 1600 LUTs in einem sinnvollen Maß zu überschreiten um daraufhin das Optimierungspotential des Entwurfs zu untersuchen
und nach und nach die Größe zu reduzieren, bis schließlich das Größenlimit eingehalten wird.
So sollten zum Beispiel nicht einfach alle 24 Runden in einem Takt über eine riesige kombinatorische Schaltung berechnet werden,
da selbst für eine einzelne Runde die 1600 LUTs für die 1600 Bit Ausgabe schon kritisch sind.
Weil sich die einzelnen Runden voneinander nur in einem Rundenindex unterscheiden, der sich in einer 64-Bit Konstanten in der Iota-Funktion manifestiert,
hätte man eine um Größenordnungen zu große Schaltung mit sehr viel repetetiver Logik.

\subsection{Aufbau}
Daher Bestand der erste Entwurf direkt aus einer kombinatorischen Schaltung, die eine komplette Runde berechnet sowie einem Register,
das die Ausgabe dieser Schaltung speichert und wieder als Eingabe bereit stellt.
[Bild hier + Beschreibung]

\subsection{Bewertung}
Letztendlich besteht der erste Entwurf aus \comment{circa 3500 LUTs, Zahl muss noch präzisiert werden} ohne Kommunikationslogik.
Die Zahl setzt sich aus ungefähr 1600 LUTS für das Speichern der Daten zusammen und der Rest wird von der kombinatorischen Rundenfunktion benötigt.
Das ist zwar noch etwa doppelt so groß wie das finale Design am Ende sein soll, aber mit ein paar Anpassungen kann die größe potentiell weiter reduziert werden.

\subsection{Optimierungsansätze}
Der wohl naheliegendste Gedanke ist den Datenspeicher zu reduzieren, da er mit seinen 1600 LUTs bereits das gesammte Budget alleine benötigt.

\subsubsection{Aufspalten der Rundenfunktion}
Um die Rundenfunktion in mehrere Teile aufteilen zu können, ist eine genauere Untersuchung der Funktion selbst notwendig um Teie ausfindig zu machen,
die unabhängig voneinander berechnet werden können. Das Aufteilen in die gleichen Teilfunktionen, aus denen sie zusammengesetzt ist, ist nicht sinnvoll.
Jede einzelne der Funktionen Theta, Rho, Pi und Chi hat genau die gleiche Eingabelänge. Es müssen also auch die Teilfunktionen selbst aufgeteilt werden.

Folgendes lässt sich aber über die einzelnen Teilfunktionen festhalten:
Theta ist eine Slice-orientierte Funktion.
Jeder Slice S(i) der Ausgabe hängt nur von zwei Slices S(i) und S(i - 1) der Eingabe ab.
Rho hingegen ist eine Lane-orientierte Funktion. Genauer genommen einfach ein Bit-Rotate jeder Lane, wobei die Weite der Rotation vom Index der Lane abhängt.
Pi und Chi sind auch Slice-orientiert, anders als Theta hängt jedoch ein Ausgabe-Slice S(i) nur von einem Einzigen Eingabe-Slice S(i) ab.
Iota ist ein Spezialfall; da sie nur eine Rundenkonstante auf eine einzige Lane aufaddiert.
Dieses Operation kann aber auch als Slice-Operation Iota(S(i), i, r) dargestellt werden, die zusätzlich von dem Slice-Index i abhängt.
Jede Teilfunktion lässt sich also Schrittweise mit einem Teil des Datenblocks berechnen, wobei Rho sich in der Hinsicht von den anderen Funktionen unterscheidet,
dass sie Lane-orientiert arbeitet. Die Rundenfunktion Rnd(r) = Iota(r) . Chi . Pi . Rho . Theta kann also in mehrere Schritte unterteilt werden,
die den Datenblock wiederum in jeweils mehreren Schritten verarbeiten. Weiterhin sollen allerdings so viele Teilfunktionen wie möglich
zusammengefasst werden können, um die benötigte Anzahl an Schritten zu minimieren. Durch Abrollen der Permutationsfunktion lässt sich diese Anzahl auf zwei reduzieren.
Rollt man die Permutationsfunktion
einmal ab, so erhält man nach Einsetzen der Definition von $Rnd_r$:
\begin{align*}
    \text{KECCAK-p} & = \bigcomp_{r = 0}^{23} Rnd_r = \bigcomp_{r = 0}^{23} \iota_r \circ \chi \circ \pi \circ \rho \circ \theta \\
    & = \iota_{23} \circ \chi \circ \pi \circ \rho \circ \theta \circ (\bigcomp_{r = 0}^{22} \iota_r \circ \chi \circ \pi \circ \rho \circ \theta) \\
    & = \iota_{23} \circ \chi \circ \pi \circ \rho \circ (\bigcomp_{r = 0}^{22} \theta \circ \iota_r \circ \chi \circ \pi \circ \rho) \circ \theta
\end{align*}

\begin{align*}
    \alpha_r & \coloneq \iota_r \circ \chi \circ \pi \\
    \beta_r & \coloneq \theta \circ \alpha_r \\
    \text{KECCAK-p} & = \alpha_{23} \circ \rho \circ (\bigcomp_{r = 0}^{22} \beta_r \circ \rho) \circ \theta
\end{align*}

\begin{align*}
    \gamma_r & = 
    \begin{cases}
        \theta & , r = -1 \\
        \alpha_{23} & , r = 23 \\
        \beta_r & , \text{sonst}
    \end{cases} \\
    \text{KECCAK-p} & = \gamma_{23} \circ \rho \circ (\bigcomp_{r = 0}^{22} \gamma_r \circ \rho) \circ \gamma_{-1} \\
    & = (\bigcomp_{r = 0}^{23} \gamma_r \circ \rho) \circ \gamma_{-1}
\end{align*}

Anmerkung: Diese Schreibweise ist deshalb äquivalent, weil Theta nicht vom Rundenindex r abhängt.
Mit Definition des Slice-orientierten Schlusses Alpha(r) = Iota(r) . Chi . Pi . Rho sowie dem Slice-orientierten Schleifenteil
Beta(r) = Theta . Alpha(r)
lässt sich die Permutationsfunktion schreiben als
KECCAK-P = Alpha(23) . (.[r = 0 ... 22] Beta(r)) . Theta
Der folgende Schritt ändert zwar rein semantisch nichts an den Teilfunktionen, erlaubt aber eine Implementierung, bei der jede Teilfunktion
genau einmal in Logik umgesetzt werden muss wie Schaubild [Schaubild Nr] zeigt.
Mit Gamma(r) =  Theta   | r = -1
				Alpha	| r = 23
				Beta(r) | sonst

[Schaubild zur Implementierung von Gamma und Delta]

Um Delta zu berechnen genügt es, zuerst schrittweise Rho zu berechnen (wenn r /= -1), wozu nur immer eine Lane benötigt wird
und danach kann Gamma schrittweise aus den Slices des Zwischenergebnisses berechnet werden.
Auf diese Weise kann die Rundenfunktion in möglichst große Teilschritte zerlegt werden, in denen die Datenabhängigkeiten der Ausgabebits möglichst gering sind.

\subsubsection{BRAM als Datenspeicher}
Ersetzt man das Flip-Flop-Register durch einen BRAM-Block, so spart man potentiell den gesammten Platz, den das Register einnimmt.
Für jedes Bit, das gleichzeitig gelesen oder geschrieben wird, sollte man jedoch mindestens ein LUT einkalkulieren,
damit nicht nur Daten von der kombinatorischen Schaltung, sondern auch von außerhalb geschrieben werden können.
Diese Idee ist daher nur dann sinnvoll, wenn auch die Rundenfunktion weiter aufgeteilt wird, sodass nicht der ganze Block auf einmal als Eingabe vorliegen muss.
Leider lässt sich dieser Ansatz nicht gut mit der vorherigen Idee verbinden, da der BRAM im Gegensatz zum Flip-Flop-Register es nicht erlaubt sowohl Slices als auch Lanes in nur einem Takt auszulesen.
[Schaubild mit Erklärung]

In Iteration 3 werden wir nochmal über diesen Ansatz weiterverfolgen, aber vorerst soll der Datenspeicher weiterhin als Flip-Flop Register implementiert werden.

\subsubsection{Aufspalten des Beschleunigers}
Da ein Atom nicht ausreicht um den ganzen Datenblock zusammen mit der Rundenfunktion zu halten, kann der Beschleuniger auch in bis zu 5 Blöcke aufgeteilt werden,
wobei jeder Block nur noch einen Teil des Datenblocks hält und nur für diesen ihm zugeteilten Teil die Rundenfunktion berechnet. Da das Ergebnis der Rundenfunktion
auch noch von den Daten anderer Blöcke abhängt, müssen diese Daten über das Interface zwischen den Atomen ausgetauscht werden. Zwei Aspekte sind bei der Aufteilung zu beachten:
\begin{enumerate}
    \item Wie viele Blöcke sind sinnvoll? Mit höherer Anzahl an Blöcken nimmt die Datenmenge ab, die jeder Block speichern muss und da jeder Block über die Implementierung der Rundenfunktion
    verfügt, kann auch die Berechnung parallel auf den Atomen durchgeführt werden. Leider steigt mit der Anzahl der Atome auch die Menge an Datenabhängigkeiten zwischen den Atomen.
    Es gilt also herauszufinden an welchem Punkt in der konkreten Architektur das Interface zum Bottleneck wird. Darüber hinaus führt die Erhöhung der Atom-Anzahl zu Leistungseinbußen.
    \item Wie werden die Daten am besten auf die Atome aufgeteilt? Die Daten müssen so auf die Atome verteilt werden, dass die Datenabhängigkeiten für die Rundenfunktion möglichst gering sind.
    Jedoch sollte das Muster auch nicht zu kompliziert sein. Für die Übertragung der Daten muss ein Kommunikationsprotokoll festgelegt werden, das bestimmt welche Teile der Daten in welchem Takt ausgetauscht werden.
    Ist das Muster zu komplex, so benötigt die Implementierung des Protokolls zu viel Platz.
\end{enumerate}

Weiterhin wäre es schön, wenn die verschiedenen Blöcke allesamt baugleich sind, es würden also alle Atome mit dem gleichen Beschleuniger beladen und für welchen Teil der Daten ein Atom verantwortlich ist,
wird ihm über einen Initialisierungsparameter mitgeteilt. Im Folgenden werden nun ein par der naheliegendsten Aufteilungsmuster untersucht:

\subsubsection{2 Block Zeilen-orthogonal}
[Schaubild]
Spaltet man die Daten wie in Schaubild [Bild Nr] gezeigt, sodass jeder Block jeweils 32 der insgesamt 64 Slices enthält, so kann jeder Block sehr einfach die Gamma-Funktion für sein Block-Tile berechnen.
Einzig die Slices 31 und 63 müssen ausgetauscht werden, da die Theta-Funktion für jeden Slice auch den benachbarten linken Slice benötigt.
Die Berechnung der Rho-Funktion ist ein wenig komplizierter. 

\subsubsection{2 Block Spalten-orthogonal}
[Schaubild]
Spaltet man die Daten entlang der Lanes, so muss für die Gamma-Funktion jeder Slice einmal zwischen den Atomen ausgetauscht werden. Die Berechnung kann allerdings weiterhin parallel erfolgen.
Auch die Rho-Funktion kann parallel berechnet werden und benötigt keinerlei Kommunikation.
Anmerkung zu Reihen-orthogonalen Mustern:
Die Gamma-Funktion benötigt ganze Slices für die Berechnung, wenn ein Slice in einem Schritt berechnet werden soll. Daher bestehen für Reihen-orthogonale Aufteilungsmuster exakt die gleichen
Vor- und Nachteile wie für Spalten-orthogonale Muster.
Einzig für das Ergebnis ist ein Spalten-orthogonales Muster vorteilhaft, da das Endergebnis aus den ersten 4 Lanes besteht und nur dort alle 4 Lanes in einem Atom enthalten sind.

\subsubsection{4 Block Muster}
[Schaubild]
Für die Aufteilung in 4 Blöcke können so die vorherigen Muster mehrfach angewendet oder auch miteinander kombiniert werden.
Der Speicheraufwand sinkt hier zwar auf etwa 25\% der gesammten Datenmenge, jedoch ist der zu erwartende Gewinn an LUTs nicht mehr so groß wie vom Schritt von 1 auf 2 Atome.
Zudem steigt der Kommunikationsaufwand deutlich an, was nicht nur eine erhöhte Ausführungszeit mit sich bringt, sondern auch wieder mehr Platz im Atom benötigt.

Für den zweiten Entwurf soll daher das 2 Block Spalten-orthogonale Muster verwendet werden, da das Kommunikationsprotokoll am wenigsten komplex ist.
Dadurch soll der Platz des Entwurfs möglichst klein gehalten werden auf Kosten der leicht höheren ausgetauschten Datenmenge und der damit verbundenen Ausführunszeit.
 

