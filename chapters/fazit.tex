In dieser Arbeit haben wir gezeigt, wie ein Hardwarebeschleuniger für eine rekonfigurierbare Prozessorplattform
wie den i-Core implementiert werden kann, die die starken Anforderungen, besonders was die maximale Größe des Beschleunigers betrifft, erfüllt.
Dazu sind wir von einem monolithischen Design mit sehr hoher Berechnungsgeschwindigkeit ausgegangen und sind
nach und nach durch kleine Änderungen zu einem modularen System gekommen, das zwar eine geringere
Berechnungsgeschwindigkeit besitzt, dafür aber die Anforderungen der Plattform erfüllt.
Für die vorgenommenen Verbesserungen haben wir verschiedene Ansätze diskutiert und damit die weiteren
Entwurfsentscheidungen begründet. Der entstandene Beschleuniger besteht aus insgesamt vier Atomen,
nutzt jedoch den Platz der den einzelnen Atomen zur Verfügung steht, nicht vollständig aus.
Daher haben wir uns noch weitere vielversprechende Verbesserungsmöglichkeiten für die Zukunft überlegt,
die den noch verfügbaren Platz verwenden, um die Berechnungsgeschwindigkeit weiter zu erhöhen.
Als Beispiel sei da die Erweiterung der Speicherschnittstelle genannt, wodurch die Berechnungzeit der
$\rho$-Funktion, die etwa 70\% der gesamten Berechnungszeit beansprucht, nochmal halbiert werden kann.
Am Ende haben wir den aktuellen Beschleuniger mit einer Software-Implementierung verglichen und festgestellt,
das der entworfene Beschleuniger die Berechnung der KECCAK-p-Funktion, welche bei SHA-3 den alleinigen Rechenaufwand darstellt,
um einen Faktor 36 schneller berechnet, als es die Software auf dem i-Core tut. Über eine genauere Untersuchung
der Berechnungsabläufe in der Software können wir schätzen, dass etwa ein Faktor 6 der gemessenen Beschleunigung
durch die veränderte Berechnungsmethode entsteht und ein weiterer Faktor 6 durch Einflüsse wie Kontrollfluss-Strukturen
in der Software und Speicherzugriffe mit langen Zugriffszeiten, durch zum Beispiel Cache Misses, entstehen.