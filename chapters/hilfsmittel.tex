Für den Entwurf und das Testen des Beschleunigers wurden verschiedene Werkzeuge verwendet.
Sie sind hier kurz aufgeführt und ihr Nutzen wird kurz erklärt. \\
Bei \textit{nvc} handelt es sich um einen Open Source Compiler und Simulator für VHDL-Designs.
Die Designs werden über den LLVM-Zwischencode direkt in Maschinencode übersetzt. Die entstehende
Simulation entspricht zwar in einigen Details nicht dem Verhalten in der Simulation des 
in Vivado integrierten Simulators, ist aber durch seine hohe Übersetzungsgeschwindigkeit
gerade für automatisierte Tests hervorragend. \\
\textit{GTKWave} ist ein sogenannter Wave Viewer. Mit ihm können die von der Simulation generierten
Signaldiagramme visualisiert werden, um Fehler im Design zu finden. \\
Mit \textit{Vivado} können die erstellten Designs für die tatsächlich verwendete Hardware synthetisiert und implementiert werden.
Es handelt sich um die hauseigene Software von Xilinx und bietet eine große Auswahl an verschiedenen Werkzeugen, die einem beim
Entwurf und beim Debugging helfen. Auch das Generieren der Bitstreams, die am Ende auf das FPGA geladen werden, können hiermit erzeugt werden. \\
Als Testplattform für vollständige Integrationstests diente ein \textit{Xilinx VC 707 Board}. Es ist das gleiche Board,
das auch für den i-Core verwendet wird. \\
Um Programme für den i-core zu compilieren, wird eine modifizierte Version des BCC-Compilers verwendet.
Der \textit{BCC-Compiler} ist auf den Leon3 zugeschnitten und um die Spezialinstruktionen des i-Core zu unterstützen,
wird eine modifizierte Version verwendet. \\
Zur Generierung der VLCWs, die vom SI Execution Controller benötigt werden, um die Spezialinstruktionen bearbeiten zu können,
wurde der \textit{BC-Assembler} verwendet. Dieser generiert die VLCWs aus einer Art Assemblersprache. \\
Mit \textit{GRMON2} können Programme auf dem i-Core ausgeführt werden. Es handelt sich um einen konsolenbasierten Debug-Monitor,
der verwendet wird, weil er den Leon3 unterstützt.