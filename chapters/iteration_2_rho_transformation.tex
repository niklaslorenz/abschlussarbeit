\subsubsection{Rho-Transformation}

Das Berechnen der Rho-Funktion auf Tile orientierten Daten kann mit Hilfe eines Schieberegisters als Puffer realisiert werden.
Schreibt man alle Rotationen, die größer als 32 Stellen sind, als Rechts-Rotationen um, so müssen in jedem Atom maximal 7 Links- und Rechts-Rotationen durchgeführt werden.
Hier bietet es sich wieder an, die Berechnung in zwei Schritte aufzuteilen. Im ersten Schritt werden die Links-Rotationen berechnet, während die anderen Lanes nicht verändert werden.
Im zweiten Schritt werden dann analog die Rechts-Rotationen durchgeführt. Um das Prinzip zu verdeutlichen, schauen wir uns zuerst die Verarbeitung einer einzelnen Lane an.
Für die Rotation um k Bits einer Lane l mit k \< 32, kann die obere Hälfte von l rechts an l angefügt werden.
w = l[63 downto 0] || l[63 downto 32]
rotl(l, k) = w[95 - k downto 31 - k]

Interessant ist, dass diese Berechnung über ein 32 Bit Schieberegister realisiert werden kann.

b[31 downto 0] := l[63 downto 32]
for i in 0 to 63 loop
    r[i] := b[32 - k]
    b := (b >> 1) | (l[i] << 63)
end loop
return r

Mit diesem Vorgehen können die Rotationen berechnet werden, wobei nicht die ganze Lane, sondern immer nur die Hälfte in einem Puffer vorliegen muss.
Da die Berechnung der Rechts-Rotationen analog funktioniert, kann ein Puffer für beide Arten von Rotation verwendet werden.
Des weiteren lässt sich die schleife auch abrollen, womit mehrere Bits auf einmal verarbeitet werden können und natürlich können mit mehreren Puffern auch
mehrere Lanes gleichzeitig rotiert werden. Dabei unterscheiden sich die einzelnen Lanes nur im Index der Auswahl ihrer Ergebnisbits.
Für den konkreten Fall reichen insgesamt 7 Puffer aus. So können im ersten Schritt alle Links- und im zweiten Schritt alle Rechts-Rotationen berechnet werden.
[Schaubild]

\subsubsection{BRAM als Datenspeicher}
Mit dem neuen Ansatz für die Berechnung der Rho-Funktion kann auch der Datenspeicher in den BRAM verschoben werden, da das Problem der unterschiedlichen Ausrichtung der benötigten Eingabedaten behoben ist.
Ein BRAM Block unterstütz dabei bis zu zwei Lese- und Schreibports. Das ist essenziell für die Berechnung der Gamma-Funkion, da durch die Pipeline in jedem Takt sowohl Daten für die eigenen Berechnungen als
auch für die Berechnungen des anderen Atoms gelesen werden müssen und auch die Ergebnisse beider Atome gleichzeitig festgehalten werden.
Da die Gamma-Funktion immer zwei Slices gleichzeitig verarbeitet, bietet sich dieses Format auch für die Speicherstruktur an. So können von jedem Port immer zwei Tiles gleichzeitig adressiert werden.
Da jeder Atom-Contrainer über insgesammt 3 BRAM Blöcke verfügt, können die Ergebnisse der Gamma-Funktion auch in einem anderen BRAM-Block gespeichert werden.
Die Rho-Funktion kann tatsächlich quasi inplace in einem BRAM-Block berechnet werden, da der BRAM read-before-write unterstützt. Wird ein Tile k gelesen und liegt das Ergebnis n Takte später vor,
so kann es an der Stelle k + n gespeichert werden, nachdem im gleichen Takt der alte Slice mit dem Index k + n gelesen wurde.
Das bedeutet, dass beide Ports gleichzeitig Daten für die Berechnung bereitstellen können, sodass immer 4 Tiles gleichzeitig in den Puffer eingelesen werden können.
Auf diese Weise benötigt die Berechnung einer vollständigen Rotation somit theoretisch etwa 8 Takte zum Füllen des Puffers mit den Initialwerten und 16 Takte zum Lesen Schreiben aller Tiles zuzüglich
zu ein paar Verzögerungstakten aufgrund des BRAMs.

\subsubsection{Datenbus}
Da sowohl Gamma als auch Rho in der Zusammenarbeit mit dem BRAM wie es oben beschrieben ist, niemals auf zwei unterschiedlichen BRAM-Blöcken gleichzeitig schreiben oder gleichzeitig lesen,
können die Datenleitungen für die beiden Speicherblöcke zusammengelegt werden.
[Schaubild + Erklärung]

