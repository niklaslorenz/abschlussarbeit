Für Vergleiche zwischen Software-Algorithmen und Hardware-Beschleunigern ist es wichtig, eine möglichst effiziente Software-Implementierung zu verwenden,
um ein aussagekräftigen Ergebnis zu erhalten. Daher wird für alle Vergleiche in Kapitel \ref{cha:ergebnisse} eine fremde Implementierung verwendet,
die viele Optimierungen enthält. Sie wird vom Github-Nutzer \textit{brainhub} \cite{sha3-impl} unter MIT-Lizenz zur Verfühgung gestellt (\ref{fig:keccak_impl_license}).
Im Folgenden sind die interessanten Segmente der KECCAK-p-Funktion aufgelistet, wobei einige Symbole zur besseren Lesbarkeit umbenannt wurden.

\begin{figure}
\lstset{language=C}
\begin{lstlisting}[label={lst:keccak_impl_license}]
MIT License

Copyright (c) 2020 brainhub

Permission is hereby granted, free of charge, to any
person obtaining a copy of this software and associated
documentation files (the "Software"), to deal in the
Software without restriction, including without limitation
the rights to use, copy, modify, merge, publish,
distribute, sublicense, and/or sell copies of the Software,
and to permit persons to whom the Software is furnished to
do so, subject to the following conditions:

The above copyright notice and this permission notice shall
be included in all copies or substantial portions of the
Software.

THE SOFTWARE IS PROVIDED "AS IS", WITHOUT WARRANTY OF ANY
KIND, EXPRESS OR IMPLIED, INCLUDING BUT NOT LIMITED TO THE
WARRANTIES OF MERCHANTABILITY, FITNESS FOR A PARTICULAR
PURPOSE AND NONINFRINGEMENT. IN NO EVENT SHALL THE AUTHORS
OR COPYRIGHT HOLDERS BE LIABLE FOR ANY CLAIM, DAMAGES OR
OTHER LIABILITY, WHETHER IN AN ACTION OF CONTRACT, TORT OR
OTHERWISE, ARISING FROM, OUT OF OR IN CONNECTION WITH THE
SOFTWARE OR THE USE OR OTHER DEALINGS IN THE SOFTWARE.
\end{lstlisting}
\caption{Lizenz der verwendeten SHA-3-Implementierung}
\label{fig:keccak_impl_license}
\end{figure}

\begin{figure}
\lstset{language=C}
\begin{lstlisting}[label={lst:keccak_impl_utils}]
static const uint64_t keccak_rount_constants[24] = {
    0x0000000000000001ULL, 0x0000000000008082ULL,
    0x800000000000808aULL, 0x8000000080008000ULL,
    0x000000000000808bULL, 0x0000000080000001ULL,
    0x8000000080008081ULL, 0x8000000000008009ULL,
    0x000000000000008aULL, 0x0000000000000088ULL,
    0x0000000080008009ULL, 0x000000008000000aULL,
    0x000000008000808bULL, 0x800000000000008bULL,
    0x8000000000008089ULL, 0x8000000000008003ULL,
    0x8000000000008002ULL, 0x8000000000000080ULL,
    0x000000000000800aULL, 0x800000008000000aULL,
    0x8000000080008081ULL, 0x8000000000008080ULL,
    0x0000000080000001ULL, 0x8000000080008008ULL
};

static const unsigned keccak_rho_distances[24] = {
    1, 3, 6, 10, 15, 21, 28, 36, 45, 55, 2, 14,
	27, 41, 56, 8, 25, 43, 62, 18, 39, 61, 20, 44
};

static const unsigned keccakf_pi_indices[24] = {
    10, 7, 11, 17, 18, 3, 5, 16, 8, 21, 24, 4, 15,
	23, 19, 13, 12, 2, 20, 14, 22, 9, 6, 1
};

#define ROTL(x, y) \
	(((x) << (y)) | ((x) >> ((sizeof(uint64_t)*8) - (y))))
#endif
\end{lstlisting}
\caption{Konstanten und Hilfsfunktionen der KECCAK-p-Implementierung}
\label{fig:keccak_impl_utils}
\end{figure}


\begin{figure}
\lstset{language=C}
\begin{lstlisting}[label={lst:keccak_impl}]
static void keccak_p(uint64_t stateArray[25]) {
  int i, j, round;
  uint64_t temp, buffer[5];
  #define KECCAK_ROUNDS 24

  for(round = 0; round < KECCAK_ROUNDS; round++) {

    /* Theta */
    for(i = 0; i < 5; i++)
      buffer[i] = stateArray[i] ^ stateArray[i + 5]
	    ^ stateArray[i + 10] ^ stateArray[i + 15]
        ^ stateArray[i + 20];

    for(i = 0; i < 5; i++) {
      temp = buffer[(i + 4) % 5] ^ ROTL(buffer[(i + 1) % 5], 1);
      for(j = 0; j < 25; j += 5)
        stateArray[j + i] ^= t;
    }

    /* Rho Pi */
    temp = stateArray[1];
    for(i = 0; i < 24; i++) {
      j = keccakf_pi_indices[i];
      buffer[0] = stateArray[j];
      stateArray[j] = ROTL(t, keccak_rho_distances[i]);
      temp = buffer[0];
    }

    /* Chi */
    for(j = 0; j < 25; j += 5) {
      for(i = 0; i < 5; i++)
        buffer[i] = stateArray[j + i];
        for(i = 0; i < 5; i++)
          stateArray[j + i] ^= (~buffer[(i + 1) % 5])
                               & buffer[(i + 2) % 5];
    }

    /* Iota */
    stateArray[0] ^= keccak_round_constants[round];
  }
}
\end{lstlisting}
\centering
\caption{Implementierung der KECCAK-p-Funktion}
\small Mit dieser Implementierung wurden die Tests in Kapitel \ref{cha:ergebnisse} durchgeführt.
\label{fig:keccak_impl}
\end{figure}

