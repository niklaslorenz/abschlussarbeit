Kryptographische Verfahren wie Verifikation digitaler Daten sind seit jeher wichtig für die Kommunikation auf unsicheren Kanälen.
Gerade vernetzte Kleinstrechner wie IoT-Geräte sind aufgrund ihrer geringen Leistungsfähigkeit sehr anfällig für Angreifer.
Verfahren zur Programmverifikation können dem entgegenwirken, benötigen jedoch verhältnismäßig viel Rechenleistung.
Mit Hilfe spezialisierter Hardware, die die nötigen Aufgaben effizienter ausführt, kann dieser zusätzliche Aufwand reduziert werden.
Rekonfigurierbare Prozessoren können hier Abhilfe schaffen. Sie erlauben es, kleine Hardwarebeschleuniger zur Laufzeit zu laden,
wenn sie benötigt werden und rechenintensive Aufgaben zu erledigen. Gleichzeitig müssen sie auch nicht für das spezielle
Einsatzgebiet angefertigt werden, sondern können nachträglich spezialisiert werden, was auch die Produktionskosten reduziert.
Des Weiteren können die zur Verfügung gestellten Beschleuniger-Blöcke zusätzlich auch für weitere einsatzspezifische Aufgaben verwendet werden.
In dieser Arbeit wollen wir daher einen Beschleuniger für einen rekonfigurierbaren Prozessor entwickeln, der die Berechnung einer
kryptographischen Hashfunktion optimiert, dem Herzstück vieler Verifizierungsalgorithmen. Dabei wollen wir uns den i-Core als Prozessorarchitektur
zunutze machen und konkret eine SHA-3-Hashfunktion für ihn implementieren. SHA-3 ist eine relativ neue Familie an Hashfunktionen,
die auch in naher Zukunft noch viel Sicherheit verspricht und viele Probleme bisheriger Hashfunktionen verbessert.
Dazu wollen wir uns zuerst einmal den i-Core, sowie SHA-3 genauer anschauen und einen Einblick in ihre Funktionsweisen erhalten.
Danach werden wir die Implementierung des Beschleunigers in einem iterativen Entwurfsverfahren betrachten. Das Ziel des
Entwurfsverfahren besteht darin, einen Beschleuniger zu entwickeln, der die Anforderungen der i-Core-Architektur erfüllt
und möglichst viel seines Leistungspotentials dabei ausschöpft. Daher werden wir uns auch verschiedene Entwürfe anschauen,
um aufzuzeigen warum einige Einschnitte in der Leistungsfähigkeit notwendig sind, um die Anforderungen des i-Core zu erfüllen
und mit welchen Ansätzen trotzdem eine möglichst hohe Effizienz beibehalten werden kann. Am Ende werden wir die verschiedenen
Ansätze untereinander vergleichen und besonders gute Aspekte herausarbeiten, sowie weitere Optimierungsansätze aufzeigen,
die die Leistungsfähigkeit noch weiter steigern können. Außerdem wollen wir den entworfenen Beschleuniger mit einer
Software-Implementierung vergleichen und bestimmen wie viel effektiver der Beschleuniger gegenüber der stumpfen Software-Berechnung ist.