Kryptographische Verfahren wie die Verifikation digitaler Daten oder Kommunikationspartnern sind seit jeher wichtig für die Kommunikation auf unsicheren Kanälen.
Gerade vernetzte Kleinstrechner wie IoT-Geräte sind aufgrund ihrer geringen Leistungsfähigkeit sehr anfällig für Angreifer \cite{iot-security}.
Verfahren zur Programmverifikation können dem entgegenwirken, benötigen jedoch verhältnismäßig viel Rechenleistung.
Mit Hilfe spezialisierter Hardware, die die nötigen Aufgaben effizienter ausführt, kann dieser zusätzliche Aufwand reduziert werden.
Rekonfigurierbare Prozessoren bieten die Möglichkeit kleine Hardwarebeschleuniger zur Laufzeit zu laden,
wenn sie benötigt werden, und rechenintensive Aufgaben zu erledigen. Gleichzeitig müssen sie auch nicht für ein spezielles
Einsatzgebiet angefertigt werden, sondern können nachträglich spezialisiert werden, was auch die Produktionskosten reduziert.
Des Weiteren können die zur Verfügung gestellten Beschleuniger-Blöcke zusätzlich auch für weitere einsatzspezifische Aufgaben verwendet werden.

In dieser Arbeit wird ein Beschleuniger für einen rekonfigurierbaren Prozessor entwickelt, der die Berechnungsgeschwindigkeit einer
kryptographischen Hashfunktion optimiert, dem Herzstück vieler Verifizierungsalgorithmen. Dabei dient der i-Core (siehe Kapitel \ref{cha:icore}) als Prozessorarchitektur
und konkret wird eine SHA-3-Hashfunktion für ihn implementiert. SHA-3 ist eine relativ neue Familie an Hashfunktionen,
die auch in naher Zukunft noch viel Sicherheit verspricht und viele Probleme bisheriger Hashfunktionen löst.

Es wird zuerst ein kleiner Einblick in den i-Core, sowie SHA-3 gegeben und
danach wird der Entwurfsprozess vorgestellt, in dem der entstandene Beschleuniger entwickelt wurde. Das Ziel war es ein Design zu entwickeln, das die Anforderungen
der i-Core-Architektur erfüllt und möglichst viel seines Leistungspotentials dabei ausschöpft. Dabei werden wir uns
verschiedene aufeinander aufbauende Entwürfe anschauen. Zu Jedem werden die getroffenen Entwurfsentscheidungen diskutiert und die
Notwendigkeit gewisser Einschnitte in der Leistungsfähigkeit zur Erfüllung der Anforderungen erläutert. Außerdem werden Ansätze
vorgestellt, mit denen die Anforderungen mit nur geringen Leistungseinbußen eingehalten werden können.

Am Ende werden die verschiedenen Ansätze untereinander verglichen und besonders gute Aspekte herausgearbeitet,
sowie weitere Optimierungsansätze aufgezeigt, die die Leistungsfähigkeit noch weiter steigern können. Außerdem wird der entworfene Beschleuniger
mit einer Software-Implementierung verglichen und der tatsächliche Speedup des Beschleuniger gegenüber der stumpfen Software-Berechnung bestimmt.