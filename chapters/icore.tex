Der icore ist ein rekonfigurierbarer Prozessor \comment{(Cite Missing)}, das bedeutet er besitzt neben den klassischen Komponenten
eines Prozessors noch eine zur Laufzeit konfigurierbare Einheit aus Hardwarebeschleunigern. 
Als Grundlage dient ein modifizierter Leon3-Kern, ein 7-Stufen-Pipeline-Prozessor basierend auf der SPARC-V8-Architektur \comment{(Cite Missing)}.
Er wird erweitert um eine sogenannte \textbf{Reconfigurable Fabric}, einen Container für fünf Beschleunigerblöcke mit Anbindung an Speicher und Registerdatei.
Wie genau diese Fabric aufgebaut ist und wie sie funktioniert schauen wir uns in Abschnitt \ref{sec:icore_arch} an.
Vorher wollen wir uns aber erstmal überlegen wofür rekonfiguruerbare Prozessoren eigentlich gut sind.
\section{Rekonfigurierbare Prozessoren}
Herkömmliche Prozessoren verfügen über einen mehr oder weniger komplexen Befehlssatz, der es ihnen erlaubt jede beliebige Berechnung durchzuführen,
indem die gewünschte Rechenoperation in mehrere vom Befehlssatz unterstützte Operationen aufgeteilt wird. Dadurch sind sie extrem flexibel.
Für rechenintensive Aufgaben, bei denen komplexere Operationen sehr oft ausgeführt werden müssen, ist dieser Ansatz jedoch nachteilig,
da der Prozessor viel Zeit benötigt, um diese komplexen Operationen zu berechnen. Um dem Abhilfe zu verschaffen, wird spezializsierte Hardware
wie zum Beispiel Grafikkarten, Netzwerkkarten oder FPGAs (siehe Abschnitt \ref{sec:fpga}) eingesetzt, die besonders effizient eine bestimmte Art von Aufgabe erfüllen.
Durch die fortschreitende Digitalisierung und Vorstöße in Bereichen wie der Industrie 4.0 oder dem Internet der Dinge (IoT) wächst der Bedarf an Kleinstrechensystemen,
die für den konkreten Anwendungszweck spezielle Aufgaben übernehmen können. Diese müssen vor allem energieeffizient, klein und günstig in der Produktion sein
und müssen dabei trotzdem auch in der Lage sein komplexere Aufgaben teilweise sogar in Echtzeit erfüllen zu können.
Ein rekonfigurierbarer Prozessor bietet hier die Vorteile der hohen Flexibilität eines normalen Prozessors und verbindet sie mit der hohen Spezialisierbarkeit von FPGAs,
indem er mehrere kleine Blöcke an vom Entwickler konfigurierbare Logikblöcke bereitstellt, die mit speziellen Prozessorinstruktionen gesteuert werden können.
Auf diese Weise können auch sehr spezielle Anwendungen von Off-The-Shelf Hardware erfüllt werden, was die Produktionskosten gegenüber Spezialanfertigungen,
sowie den Energieverbrauch und den Bedarf an Rechenleistung gegenüber klassischen Prozessoren reduziert.

\subsection{FPGA}
\label{sec:fpga}
\textbf{Field Programmable Gate Arrays} (FPGAs) sind integrierte Schaltkreise, die von sich aus keine genaue Funktion implementieren. Stattdessen muss erst eine Schaltung "geladen" werden.
Auf der kleinsten Ebene bestehen sie aus Logikblöcken und Verbindungsstrukturen. In einen Logikblock kann von außen eine boolsche Funktion "einprogrammiert" werden. Dabei macht man sich zu Nutze,
dass eine n-stellige boolsche Funktion durch einen $2^n$-Bit Vektor codiert werden kann. In Abb. \comment{\ref{fig:fpga_lut}} ist zum Beispiel die Funktion $f = A \vee (B \wedge \neg C)$ dargestellt.
Der 8-Bit Ergebnisvektor wird in SRAM-Speicherzellen gehalten und durch einen 8-zu-1-Multiplexer kann mit $A$, $B$ und $C$ das entsprechende Ergebnisbit ausgewählt werden. Diese Schaltung wird Lookup Table (LUT) genannt.
Eine Logikzelle enthält neben einem solchen Lookup Table meist noch ein Flip-Flop, sodass das Ergebnis gespeichert oder direkt ausgegeben werden kann.


\section{icore-Architektur}
\label{sec:icore_arch}

\section{Erweiterung Dynamic Execution}