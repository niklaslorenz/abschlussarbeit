\chapter{Einleitung}
Kryptographische Verfahren wie die Verifikation digitaler Daten oder Kommunikationspartnern sind seit jeher wichtig für die Kommunikation auf unsicheren Kanälen.
Gerade vernetzte Kleinstrechner wie IoT-Geräte sind aufgrund ihrer geringen Leistungsfähigkeit sehr anfällig für Angreifer \cite{iot-security}.
Verfahren zur Programmverifikation können dem entgegenwirken, benötigen jedoch verhältnismäßig viel Rechenleistung.
Mit Hilfe spezialisierter Hardware, die die nötigen Aufgaben effizienter ausführt, kann dieser zusätzliche Aufwand reduziert werden.
Rekonfigurierbare Prozessoren bieten die Möglichkeit kleine Hardwarebeschleuniger zur Laufzeit zu laden,
wenn sie benötigt werden, und rechenintensive Aufgaben zu erledigen. Gleichzeitig müssen sie auch nicht für ein spezielles
Einsatzgebiet angefertigt werden, sondern können nachträglich spezialisiert werden, was auch die Produktionskosten reduziert.
Des Weiteren können die zur Verfügung gestellten Beschleuniger-Blöcke zusätzlich auch für weitere einsatzspezifische Aufgaben verwendet werden.

In dieser Arbeit wird ein Beschleuniger für einen rekonfigurierbaren Prozessor entwickelt, der die Berechnungsgeschwindigkeit einer
kryptographischen Hashfunktion optimiert, dem Herzstück vieler Verifizierungsalgorithmen. Dabei dient der i-Core (siehe Kapitel \ref{cha:icore}) als Prozessorarchitektur
und konkret wird eine SHA-3-Hashfunktion für ihn implementiert. SHA-3 ist eine relativ neue Familie an Hashfunktionen,
die auch in naher Zukunft noch viel Sicherheit verspricht und viele Probleme bisheriger Hashfunktionen löst.

Es wird zuerst ein kleiner Einblick in den i-Core, sowie SHA-3 gegeben und
danach wird der Entwurfsprozess vorgestellt, in dem der entstandene Beschleuniger entwickelt wurde. Das Ziel war es ein Design zu entwickeln, das die Anforderungen
der i-Core-Architektur erfüllt und möglichst viel seines Leistungspotentials dabei ausschöpft. Dabei werden wir uns
verschiedene aufeinander aufbauende Entwürfe anschauen. Zu Jedem werden die getroffenen Entwurfsentscheidungen diskutiert und die
Notwendigkeit gewisser Einschnitte in der Leistungsfähigkeit zur Erfüllung der Anforderungen erläutert. Außerdem werden Ansätze
vorgestellt, mit denen die Anforderungen mit nur geringen Leistungseinbußen eingehalten werden können.

Am Ende werden die verschiedenen Ansätze untereinander verglichen und besonders gute Aspekte herausgearbeitet,
sowie weitere Optimierungsansätze aufgezeigt, die die Leistungsfähigkeit noch weiter steigern können. Außerdem wird der entworfene Beschleuniger
mit einer Software-Implementierung verglichen und der tatsächliche Speedup des Beschleuniger gegenüber der stumpfen Software-Berechnung bestimmt.

\chapter{i-Core}

\section{Einführung Rekonfigurierbare Prozessoren}

\subsection{FPGA}

\section{Architektur}

\section{Erweiterung Dynamic Execution}

\chapter{SHA-3}
Sha-3 ist eine vom US-amerikanischen "National Institute of Standards and Technology" definierte Familie von kryptographischen Hashfunktion.
Dazu wird die Eingabe mithilfe eines \textbf{Paddings} in mehere gleich große Blöcke aufgeteilt. Die Blöcke werden nacheinander über eine \textbf{Schwammkonstruktion}
zu einem 1600 Bit breiten Bitvektor kombiniert, woraus dann der endgültige Hash ausgelesen wird. 

\newcommand{\bigcomp}{%
  \DOTSB
  \mathop{\vphantom{\sum}\mathpalette\bigcomp@\relax}%
  \slimits@
}
\newcommand{\bigcomp@}[2]{%
  \begingroup\m@th
  \sbox\z@{$#1\sum$}%
  \setlength{\unitlength}{0.9\dimexpr\ht\z@+\dp\z@}%
  \vcenter{\hbox{%
    \begin{picture}(1,1)
    \bigcomp@linethickness{#1}
    \put(0.5,0.5){\circle{1}}
    \end{picture}%
  }}%
  \endgroup
}
\newcommand{\bigcomp@linethickness}[1]{%
  \linethickness{%
      \ifx#1\displaystyle 2\fontdimen8\textfont\else
      \ifx#1\textstyle 1.65\fontdimen8\textfont\else
      \ifx#1\scriptstyle 1.65\fontdimen8\scriptfont\else
      1.65\fontdimen8\scriptscriptfont\fi\fi\fi 3
  }%
}

\section{Keccak-Permutation}
Als Grundlage für SHA-3 dient eine Instanz der Keccak-Permutationsfamilie KECCAK-f.
Für eine kryptographische Sicherheitsanalyse betrachtet man in der Regel das asymptotische Verhalten der erwarteten Laufzeit eines Angreifers.
Hierzu benötigt man eine Funktion mit variablem Sicherheitsparameter, damit diese Analyse durchgeführt werden kann.
Wir interessieren uns hier allerdings nur für die konkrete Instanz mit einem festen Sicherheitsparameter, die vom SHA-3 Standard[Cite missing] festgelegt wird, genauer für $SHA3-256$.
Die genaue Definition mit variablem Sicherheitsparameter ist zum Nachlesen auch in [Cite missing] angegeben.
Wir wollen uns nun zuerst einmal den Zustandsvektor ein wenig genauer anschauen, bevor wir uns die fünf Unterfunktionen ansehen, aus denen die KECCAK-f Permutation aufgebaut ist.

\subsection{Zustandsvektor}
Die KECCAK-f Permutation arbeitet auf einem 1600 Bit breiten Zustandsvektor, auch \textbf{State Array} genannt.
Am besten lässt er sich als dreidimensionale Struktur der Form 5x5x64 Bit vorstellen, siehe Abb. \ref{fig:statearray}.

\begin{figure}
	\center
	\includegraphics{images/StateArrayBeschreibung.pdf}
	\caption{Blockrepräsentation des State Array}
	\label{fig:statearray}
\end{figure}

Die Konvertierung eines eindimensionalen Vektors \textbf{V} in dieses dreidimensionale State Array \textbf{A} funktioniert wie folgt:
\begin{align*}
	\textbf{A}[x][y][z] \coloneq \textbf{V}[64(5y + x) + z] & \forall x,y = 0,...,4; z = 0,...,63
\end{align*}
Die Lanes werden also der Reihe nach erst in x-Richtung und dann in y-Richtung mit dem Inhalt von \textbf{V} gefüllt.

\subsection{Konventionen}
\begin{align*}
    \textbf{A},\ \textbf{A}^\prime & : \text{ State Array}, \\
    \textbf{B},\ \textbf{C} & : \text{ Ein Vektor aus 5 Lanes (von 0 bis 4)}, \\
    r & : \text{ Ein Rundenindex (von 0 bis 23)}
\end{align*}

\subsection{Theta-Unterfunktion}
\begin{figure}
    \center
    \includegraphics{images/theta.pdf}
    \caption{Spaltensummierung der $\theta$-Funktion}
    \label{fig:definition_theta}
\end{figure}
Die erste Unterfunktion der Keccak-Permutation ist Theta ($\theta$). Sie modifiziert jedes Bit eines State Arrays, indem sie zwei benachbarte Spalten auf das Bit aufsummiert, siehe Abb. \ref{fig:definition_theta}.
\begin{align*}
    \theta (\textbf{A}) & \coloneq \textbf{A}^\prime \text{ mit } \\
    \textbf{C}[x] & \coloneq \textbf{A}[x][0] \oplus ... \oplus \textbf{A}[x][4] && \forall x = 0,...,4 \\
    \textbf{D}[x] & \coloneq \textbf{C}[(x - 1) mod\ 5] \oplus rotr(C[(x + 1) mod\ 5], 1)\ && \forall x = 0,...,4 \\
    \textbf{A}^\prime[x][y] & \coloneq \textbf{A}[x][y] \oplus \textbf{D}[x]\ && \forall x = 0,...,4;\ y = 0,...,4
\end{align*}
Man beachte, dass hier die State Arrays nur mit x und y indiziert werden, alle Operationen finden also immer auf ganzen Lanes gleichzeitig statt.
Ein 64-Bit Prozessor kann also Theta mit Hilfe von ein paar bitweisen XOR-Operationen auf 64-Bit Operanden berechnen.
\textbf{C} dient hierzu als Zwischenspeicher für Spaltensummen und \textbf{D} wird verwendet, um jeweils zwei Spaltensummen aufzuaddieren.

\subsection{Rho-Unterfunktion}
\begin{figure}
    \center
    \includegraphics{images/rho.pdf}
    \caption{Rotationsdistanzen der Lanes für $\rho$}
    \label{fig:definition_rho}
\end{figure}
Bei der Rho-Unterfunktion ($\rho$) handelt es sich um eine einfache Bitrotation der einzelnen Lanes.
Bis auf die Lane bei $x=0 und y=0$ werden alle Lanes um eine konstante Distanz nach links rotiert.
Die genauen Distanzen sind in Abb. \ref{fig:definition_rho} aufgeführt.
\begin{align*}
    \rho (\textbf{A}) & \coloneq \textbf{A}^\prime \text{ mit } \\
    \textbf{A}^\prime[x][y] & \coloneq rotl(\textbf{A}[x][y], d[x][y])\ \forall x = 0,...,4;\ y = 0,...,4 \\
    d & :\text{ Eine 5x5 Matrix an Konstanten, siehe Abb. \ref{fig:definition_rho}}
\end{align*}

\subsection{Pi-Unterfunktion}
\begin{figure}
    \center
    \includegraphics{images/pi.pdf}
    \caption{Visualisierung der $pi$-Permutation}
    \label{fig:definition_pi}
\end{figure}
Di Pi-Unterfunktion ($\pi$) permutiert die Lanes eines State Arrays untereinander nach einer einfachen Vorschrift.
\begin{align*}
    \pi (\textbf{A}) & \coloneq \textbf{A}^\prime \text{ mit } \\
    \textbf{A}^\prime[x][y] & \coloneq \textbf{A}[(x + 3y)\ mod\ 5][x]\ \forall x = 0,...,4;\ y = 0,...,4
\end{align*}

\subsection{Chi-Unterfunktion}

\begin{align*}
    \chi (\textbf{A}) & \coloneq \textbf{A}^\prime \text{ mit } \\
    \textbf{A}^\prime[x][y] & \coloneq \textbf{A}[x][y] \oplus ((\sim \textbf{A}[(x + 1)\ mod\ 5][y]) * \textbf{A}[(x + 2)\ mod\ 5][y])\ & \forall x = & 0,...,4;\\
    && y = & 0,...,4
\end{align*}

\subsection{Iota-Unterfunktion}

\begin{align*}
    \iota (\textbf{A}, r) & \coloneq \textbf{A}^\prime \text{ mit } \\
    \textbf{A}^\prime[x][y] & \coloneq
    \begin{cases}
        \textbf{A}[x][y] \oplus C[r], & x = 0 \text{ und } y = 0 \\
        \textbf{A}[x][y], & \text{sonst}
    \end{cases} \\
    \iota_r (\textbf{A}) & \coloneq \iota(\textbf{A}, r)
\end{align*}

\subsection{KECCAK-p}

\begin{align*}
    \text{keccak-p} (\textbf{A}, r_c) & \coloneq (\iota_r \circ \chi \circ \pi \circ \rho \circ \theta)(\textbf{A}) \\
    \text{keccak-p}_r(\textbf{A}) & \coloneq \text{keccak-p} (\textbf{A}, r)
\end{align*}

\subsection{KECCAK-f}

\begin{align*}
    \text{keccak-f} (\textbf{A}) \coloneq (\bigcomp_{i = 0}^{23} \text{keccak-p}_{i})(\textbf{A})
\end{align*}

\section{Padding-Funktion $pad10^*1$}
Um eine Eingabe beliebiger Länge gescheit verarbeiten zu können, wird eine Padding-Funktion verwendet.
Diese nimmt eine Eingabe beliebiger Länge entgegen und hängt ein einfaches dynamisches Datenmuster an,
sodass die Ausgabelänge ein Vielfaches der gewünschten Blocklänge ist.
SHA-3 verwendet als Padding die Funktion $pad10^*1$. Diese hängt, wie der Name schon vermuten lässt,
zwei 1-Bits an die Ausgabe an und fügt dazwischen so viele Nullen ein, bis die Ausgabe die gewünschte Länge besitzt.

\begin{align*}
	&\begin{alignedat}[t]{2}
	\text{Für} & \text{ eine Eingabe } & M & \in \{0,1\}^*, \\
	& \text{ eine verlangte Blocklänge } & r & \in \mathbb{N} \\
	\end{alignedat} \\
	& \text{ist } pad10^*1(r, M) \coloneq M \mathbin\Vert 1 \mathbin\Vert 0^{(-|M| - 2) mod\ r} \mathbin\Vert 1
\end{align*}

\section{Schwammkonstruktion}
	\begin{figure}
		\center
		\includegraphics[scale=0.175]{images/Schwammkonstruktion.png}
		\caption{Aufbau der Schwammkonstruktion [Cite missing]}
		\label{fig:schwammkonstruktion}
	\end{figure}
	Um beliebig lange Eingaben zu einem kurzten Hashwert komprimieren zu können, verwendet SHA-3 die sogenannte Schwammkonstruktion.
	Sie erlaubt es eine Eingabe beliebiger Länge auf eine Ausgabe einer beliebigen anderen Länge d abzubilden.
	Dazu wird die Eingabe wie in Abb \ref{fig:schwammkonstruktion} abgebildet erst mit Hilfe einer Padding-Funktion in mehrere gleich große Blöcke einer festgelegten Länge abgebildet.
	Die eingabeblöcke werden dann der Reihe nach vom Schwamm "absorbiert". Danach werden auf ähnliche Weise die Ausgabeblöcke aus dem Schwamm "ausgequetscht".
	Diese Ausgabeblöcke werden dann zur finalen Ausgabe zusammengesetzt. Wenn die Ausgabelänge kein Vielfaches der Blocklänge sein sollte, wird der Rest einfach abgeschnitten.
	Der genaue Definition der Schwammkonstruktion über einer Transformation $f$ mit einer Paddingfunktion $pad$ sieht folgendermaßen aus:
	
	\begin{align*}
		&\begin{alignedat}[t]{2}
			Seien\ & n \in \mathbb{N} && \text{ die Transformationsbreite}, \\
			& r \in \{1,...,n\} && \text{ die Blocklänge}, \\
			& M \in \{0,1\}^* && \text{ die zu verarbeitende Nachricht}, \\
			& m \in \mathbb{N} && \text{ die Anzahl an Blöcken, in die M eingeteilt wird}, \\
			& d \in \mathbb{N} && \text{ die gewünschte Ausgabe}, \\
			& l \coloneq \lceil \frac{d}{r} \rceil && \text{ die benötigte Blockanzahl an Ausgabe}, \\
			& pad: \mathbb{N}x\{0,1\}^* \to (\{0,1\}^r)^+ && \text{ eine Padding-Funktion}, \\
			& f: \{0,1\}^n \to \{0,1\}^n && \text{ eine Transformation}.
		\end{alignedat} \\
		\\
		&\text{Dann ist } SPONGE[f,pad](N,d)\ \text{definiert als:} \\
		&\begin{alignedat}[t]{3}
			SPONGE[f,pad](M, d) & \coloneq \mathbf{Z}[0] \mathbin\Vert ... \mathbin\Vert \mathbf{Z}[d-1]\ mit \\
			M_0...M_{m-1} & \coloneq pad(r, M) \\
			\mathbf{A}_0 & \coloneq 0^n \\
			\mathbf{A}_i & \coloneq f(\mathbf{A}_{i-1} \oplus (M_{i-1} \mathbin\Vert 0^c))\ & \forall i = 1,...,m \\
			\mathbf{B}_1 & \coloneq \mathbf{A}_{m} \\
			\mathbf{B}_i & \coloneq f(\mathbf{B}_{i-1})\ & \forall i = 2,...,l \\
			\mathbf{Z}_i & \coloneq \mathbf{B}_i[0] \mathbin\Vert ... \mathbin\Vert \mathbf{B}_i[r-1] \\
			\mathbf{Z} & \coloneq \mathbf{Z}_1 \mathbin\Vert ... \mathbin\Vert \mathbf{Z}_l
		\end{alignedat}
	\end{align*}

\section{Sicherheitseigenschaften}

\subsection{Kollisionsresistenz}

\subsection{Post-Quantum Sicherheit der Schwammkonstruktion}

\section{Software-Implementierung}

\chapter{Implementierung}
\section{Vorgehensweise}
Ziel ist es, einen Beschleuniger zu entwickeln, der die Berechnung von SHA-3 möglichst effizient durchführt.
Da alle SHA3-Funktionen die gleiche Instanz von KECCAK-p verwenden, ist die Wahl der konkret zu implementierenden
SHA3-Funktion praktisch egal, sie unterscheiden sich nur in der Größe der Eingabeblöcke, die mit dem internen
State Array kombiniert werden. Daher legen wir uns hier auf SHA3-256 fest, die der Beschleuniger implementieren soll.
Dabei müssen aber alle Voraussetzungen der gewählten Architektur eingehalten werden. Im Falle des icore sind
das die Größenbeschränkung von maximal fünf Atomen mit jeweils 1600 LUTs, sowie das Implementieren der
Kommunikationsschnittstelle der Atome, sodass sie sowohl untereinander, als auch mit dem icore, Daten austauschen können.
Da quasi der gesamte Rechenaufwand von SHA-3 aus der wiederholten Berechnung der KECCAK-p-Funktion besteht,
soll der hier entworfene Beschleuniger genau diese KECCAK-p-Funktion berechnen.
Für den Entwurf verwenden wir hier ein iteratives Design-Verfahren. Angefangen wird mit einem Beschleuniger,
der die Architekturbeschränkungen komplett ignorieren darf, um eine maximale Berechnungsgeschwindigkeit zu erzielen.
Der Platz des Beschleunigers sollte dennoch sinnvoll genutzt werden. Auf diesem Entwurf aufbauend können dann
Strategien entwickelt werden, um die Berechnung so aufzuteilen, dass der entstehende Beschleuniger nach und nach die
Voraussetzungen der Architektur erfüllt.
\section{Erster Entwurf}
\subsection{Entwurfsziele}
Der triviale Ansatz, die KECCAK-p-Funktion direkt in einem riesigen kombinatorischen Netz zu berechnen ist zwar theoretisch am schnellsten,
jedoch wenig platzeffizient. Da die KECCAK-p-Funktion aus 24 nahezu identischen Runden aufgebaut ist, müsste auch jede einzelne
ihrer Berechnungen nebeneinander realisiert werden. Da Ein- und Ausgabedaten mit jeweils 1600 Bits in der gleichen Größenordnung liegen
wie die Größenbeschränkung des icore, ist es deutlich sinnvoller, die Berechnung der Runden nacheinander über dieselbe Realisierung der Rundenfunktion durchzuführen.
Ziel dieses Entwurfs ist daher die Implementierung der Rundenfunktion in einem rein kombinatorischen Netz, womit dann in jedem Takt
jeweils eine Runde berechnet werden kann, bis schließlich die komplette KECCAK-p-Funktion berechnet ist. Die Kommunikationsschnittstelle
der Atome soll auch rudimentär implementiert werden, sodass das Einlesen und Ausgeben von Daten den dafür vorgesehenen 64-Bit Datenkanal nutzt
und die Steuerung des Atoms über maximal 6 Kontrollbits durchgeführt wird.

\subsection{Aufbau}
\begin{figure}
    \center
    \includegraphics{images/Iteration_1.pdf}
    \caption{Aufbau des ersten Entwurfs}
    \label{fig:aufbau_iteration_1}
\end{figure}
\begin{figure}
    \center
    \includegraphics{images/Iteration_1_Speicher.pdf}
    \caption{Speicherzelle des ersten Entwurfs}
    \label{fig:speicher_iteration_1}
\end{figure}
In Abbildung \ref{fig:aufbau_iteration_1} ist der grobe Aufbau des Beschleunigers zu sehen. Die Rnd-Komponente berechnet aus den im Speicher liegenden Daten genau die Rundenfunktion.
Welche Runde sie berechnen soll, wird vom Zustandsautomaten bestimmt. Das Ergebnis kann nun in den Speicher zurückgeschrieben oder ausgegeben werden,
wobei der Selektor immer jeweils eine Lane des Ergebnisses an die Ausgabe anlegt. Der Zustandsautomat ist bei der Ausgabe dafür verantwortlich,
dass der Selektor alle Lanes nacheinander auswählt und somit das gesamte Ergebnis ausgibt. Der Speicher besteht aus 1600 Speicherzellen,
sodass jedes Bit der Ein-/Ausgabe der Rundenfunktion darin gespeichert werden kann. Jenachdem, ob das Ergebnis der Rundenfunktion oder die Eingabe
in den Beschleuniger übernommen werden soll, muss das entsprechende Bit ausgewählt werden. In Abbildung \ref{fig:speicher_iteration_1} ist eine Speicherzelle
mit dieser Auswahllogik skizziert. Aus einem State Array $\textbf{A}$ wird das Bit $\textbf{A}[x][y][z]$ in der Speicherzelle $D_{i,z}$ mit $i = 5 * y + x$ gespeichert.
Der Index i beschreibt also in welcher Lane, und der Index z an welcher Position innerhalb der Lane sich das Bit befindet. Die Kontrollsignale $w_i$ und $w_r$ bestimmen,
ob ein Bit aus der Eingabe ($I_z$) oder aus dem Ergebnis von Rnd ($R_{i,z}$) übernommen werden soll. In der Auswahllogik (blau) wird das entsprechende Bit ausgewählt und in der
Kontrolllogik (rot) wird am nächsten Takt der Wert in die Speicherzelle geschrieben, falls eines der beiden Bits übernommen werden soll.
Die Kontrolllogik muss dabei für jede Lane nur einmal realisiert werden, die Auswahllogik allerdings für jedes der 1600 Bits. 

\subsection{Bewertung}
Der gesamte Entwurf besteht aus \comment{3314 LUTs}. Davon werden etwa 1600 LUTs für den Datenspeicher verwendet.
Der Rest wird fast vollständig für die Berechnung der Rundenfunktion benötigt. Der Zustandsautomat und der Ergebnisselektor sind da verhältnismäßig klein.
Insgesamt ist der Entwurf damit ungefähr doppelt so groß wie es die Architektur des icore vorgibt. Im Folgenden wollen wir uns daher ein paar Verbesserungen anschauen,
mit denen die Größe des Beschleunigers angepasst werden kann.

\subsection{Optimierungsansätze}
\label{cha:iteration_1_optimierungen}
Die naheliegendste Möglichkeit zur Verbesserung des Platzbedarfs besteht darin den Speicheraufwand zu reduzieren, da dieser zwar viel Platz einnimmt,
aber an der Berechnung selbst nicht teilnimmt. Das Speichern der Daten außerhalb des Beschleunigers ist alleine betrachtet leider keine Option, da zur Berechnung der Rundenfunktion
ja der Gesamte Datenblock im Atom wieder vorliegen muss. Um mit weniger Daten auf einmal arbeiten zu können, muss daher zuerst die Rundenfunktion in kleinere
Teiloperationen aufgeteilt werden.

\subsubsection{Aufspalten der Rundenfunktion}
\label{cha:iteration_1_modr}
Um die Rundenfunktion in mehrere Operationen aufteilen zu können, ist eine genauere Untersuchung der Funktion selbst notwendig, um Teie ausfindig zu machen,
die unabhängig voneinander berechnet werden können. Ziel ist es, möglichst große Abschnitte in der Berechnung zu finden, die die gleiche (oder zumindest sehr ähnliche)
Operation unabhängig voneinander auf verschiedenen Daten berechnen. Diese Berechnungen können dann sequenziell statt parallel berechnet werden,
wodurch der benötige Platz reduziert wird. Damit fällt die Aufteilung in die Teilfunktionen aus der Definition der Rundenfunktion $\theta$, $\rho$, $\pi$, $\chi$ und $\iota_r$
weg, da diese Teile direkt voneinander abhängen und nicht gleichzeitig berechnet werden, sondern nacheinander.
Allerdings lassen sich alle fünf Teilfunktionen in zwei Kategorien einteilen:
\begin{enumerate}
    \item Slice-orientierte Funktionen
        Jeder Slice der Ausgabe hängt nur von einem oder zwei Slices der Eingabe ab, die anderen Slices werden ignoriert. Das sind genau die Funktionen,
        bei denen in der Definition (siehe \ref{cha:sha3_unterfunktionen}) der z-Index nicht groß verändert wird, also $\theta$, $\pi$, $\chi$ und $\iota_r$.
    \item Lane-orientierte Funktionen
        Analog handelt es sich hier um die Funktionen, bei denen jede Lane der Ausgabe nur von einer Lane der Eingabe abhängt.
        Für die Berechnung gemäß der Definition bedeutet das, dass die Indizes x und y nicht manipuliert werden werden.
        Zu den Lane-orientierten Funktionen zählen nur $\rho$ und $\iota_r$.
\end{enumerate}
$\iota_r$ zählt hierbei in beide Kategorien, da es sich nur um die Aufaddierung einer Konstanten handelt und somit jedes Bit der Ausgabe nur von einem Bit und einer Konstanten abhängt.
Die Rundenfunktion kann so in verschiedene Abschnitte eingeteilt werden, die nur aus Slice-orientierten oder Lane-orientierten Teilfunktionen bestehen.
Um eine möglichst gute Ausführungsgeschwindigkeit beizubehalten, ist es wichtig, die Anzahl dieser Abschnitte möglichst gering zu halten,
da so in jedem Abschnitt ein möglichst großer Teil der Berechnung durchgeführt wird. Die Rundenfunktion $Rnd_r$ besitzt drei dieser Abschnitte.
\begin{align*}
    Rnd_r & = \underbrace{\iota_r \circ \chi \circ \pi}_\text{Slice-orientiert} \circ \underbrace{\rho}_\text{Lane-orientiert} \circ \underbrace{\theta}_\text{Slice-orientiert}
\end{align*}
Für die Berechnung von KECCAK-p lässt sich die Rundenfunktion leicht modifizieren, sodass sie nur noch zwei Abschnitte besitzt.
Die Idee ist dabei, dass die Berechnung von $\theta$ jeweils an das Ende der vorherigen Runde verschoben wird.
Durch Einsetzen der Definition von $Rnd_r$ in KECCAK-p erhält man:
\begin{align*}
    \text{KECCAK-p} & = \bigcomp_{r = 0}^{23} Rnd_r = \bigcomp_{r = 0}^{23} \iota_r \circ \chi \circ \pi \circ \rho \circ \theta \\
    & = \iota_{23} \circ \chi \circ \pi \circ \rho \circ \theta \circ (\bigcomp_{r = 0}^{22} \iota_r \circ \chi \circ \pi \circ \rho \circ \theta) \\
    & = \iota_{23} \circ \chi \circ \pi \circ \rho \circ (\bigcomp_{r = 0}^{22} \theta \circ \iota_r \circ \chi \circ \pi \circ \rho) \circ \theta
\end{align*}
KECCAK-p lässt sich dann mit der modifizierten Rundenfunktion ($RMod_r$) wieder kompakt zusammenfassen:
\begin{align*}
    \alpha_r & \coloneq \iota_r \circ \chi \circ \pi \\
    \beta_r & \coloneq \theta \circ \alpha_r \\
    \gamma_r & \coloneq
    \begin{cases}
        \theta & , r = -1 \\
        \alpha_{23} & , r = 23 \\
        \beta_r & , \text{sonst}
    \end{cases} \\
    RMod_r & \coloneq 
    \begin{cases}
        \gamma_{-1} & , r = -1 \\
        \gamma_r \circ \rho & , sonst
    \end{cases} \\
    \text{KECCAK-p} & = \iota_{23} \circ \chi \circ \pi \circ \rho \circ (\bigcomp_{r = 0}^{22} \theta \circ \iota_r \circ \chi \circ \pi \circ \rho) \circ \theta \\
    & = \alpha_{23} \circ \rho \circ (\bigcomp_{r = 0}^{22} \beta_r \circ \rho) \circ \theta \\
    & = \gamma_{23} \circ \rho \circ (\bigcomp_{r = 0}^{22} \gamma_r \circ \rho) \circ \gamma_{-1} \\
    & = (\bigcomp_{r = 0}^{23} \gamma_r \circ \rho) \circ \gamma_{-1} \\
    & = \bigcomp_{i = -1}^{23} RMod_r
\end{align*}
Der Vorteil von $RMod_r$ gegenüber $Rnd_r$ ist wie anfangs motiviert, dass $RMod_r$ für jeden Index $r$ nur zwei Abschnitte mit unterschiedlicher Orientierung besitzt ($\rho$ und $\gamma$).
Da $\gamma$ eine Slice-orientierte Funktion ist, können alle Slices unabhängig voneinander nacheinander oder parallel berechnet.
Auf diese Weise ist es möglich, den von $\gamma$ benötigten Platz auf Kosten von mehr Berechnungszeit sehr stark zu reduzieren.
In Abbildung \ref{fig:gamma_berechnung} wird skizziert, wie die Berechnung von Gamma auf einzelnen Slices sehr platzeffizient implementiert werden kann, da nicht $\alpha$, $\beta$ und $\theta$
alle vollständig implementiert werden müssen. Über zwei Multiplexer können alle drei Teile berechnet werden.
\begin{figure}
    \center
    \includegraphics{images/gamma_berechnung.pdf}
    \caption{Implementierung der Gamma-Funktion}
    Die Funktionen $\pi_s$, $\chi_s$ und $\iota_{r,s}$ berechnen jeweils genau zwei Slices, sodass $\theta_s$ daraus einen Slice berechnen kann.
    \label{fig:gamma_berechnung}
\end{figure}

\subsubsection{BRAM als Datenspeicher}
Um nun den Platzbedarf des Speichers zu reduzieren, gibt es mehrere Möglichkeiten. Eine davon besteht darin, die Daten nicht in einem Flip-Flop-Register zu speichern,
sondern einen BRAM zur Speicherung von Daten zu verwenden. Leider lässt sich dieser Ansatz nicht direkt mit der Aufteilung der Rundenfunktion, wie oben beschrieben, kombinieren.
Grund dafür ist die unterschiedliche Orientierung der Operationen, die auf den Daten durchgeführt werden.
Aus einem Flip-Flop-Register können jederzeit beliebige Bits ausgelesen werden, sodass sowohl Slice- als auch Lane-orientierte Operationen direkt
aus dem Speicher mit Daten versorgt werden können. Möchte man den BRAM als Datenspeicher verwenden, so muss für den Speicher eine Orientierung festgelegt werden.
Um dann Operationen mit einer anderen Orientierung durchführen zu können, müsste jede Speicherstelle des BRAM nacheinander ausgelesen
und daraus das gewünschte Datenobjekt zusammengesetzt werden.
Daher wollen wir im zweiten Entwurf diesen Ansatz nicht verwenden, werden ihn aber im dritten Entwurf nochmal weiterverfolgen.

\subsubsection{Aufspalten des Beschleunigers}
Da ein Atom nicht ausreicht um den ganzen Datenblock zusammen mit der Rundenfunktion zu halten, kann der Beschleuniger auch in bis zu 5 Atome aufgeteilt werden,
wobei jeder Block nur noch einen Teil des Datenblocks hält und auch nur für einen Teil der Daten die modifizierte Rundenfunktion aus \ref{cha:iteration_1_modr} berechnet.
Da das Ergebnis der Rundenfunktion im allgemeinen auch noch von den Daten anderer Blöcke abhängt, müssen diese Daten über das Interface zwischen den Atomen ausgetauscht werden.
Zwei Aspekte sind daher bei der Aufteilung wichtig:
\begin{enumerate}
    \item Wie viele Blöcke sind sinnvoll? Mit höherer Anzahl an Blöcken nimmt die Datenmenge ab, die jeder Block speichern muss und da jeder Block über die Implementierung der Rundenfunktion
    verfügt, kann auch die Berechnung parallel auf den Atomen durchgeführt werden. Leider steigt mit der Anzahl der Atome auch die Menge an Datenabhängigkeiten zwischen den Atomen.
    Es gilt also herauszufinden an welchem Punkt die Kommunikationsschnittstelle zwischen den Atomen zum Bottleneck wird.
    \item Wie werden die Daten am besten auf die Atome aufgeteilt? Die Daten müssen so auf die Atome verteilt werden, dass die Datenabhängigkeiten für die Rundenfunktion möglichst gering sind.
    Jedoch sollte das Muster auch nicht zu kompliziert sein. Für die Übertragung der Daten muss ein Kommunikationsprotokoll festgelegt werden, das bestimmt,
    welche Teile der Daten in welchem Takt ausgetauscht werden. Ist das Muster zu komplex, so benötigt die Implementierung des Protokolls zu viel Platz.
\end{enumerate}
Weiterhin wäre es schön, wenn die verschiedenen Blöcke allesamt symmetrisch, also baugleich sind, es würden also alle Atome mit dem gleichen Beschleuniger konfiguriert
und über ein Kontrollsignal wird am Anfang der Ausführung bestimmt für welchen Teil der Daten ein Atom verantwortlich ist.
Dadurch ist der Beschleuniger leichter zu testen. Im Folgenden werden wir uns ein paar der naheliegendsten Aufteilungsmuster anschauen.

\subsubsection{2-Block Lane-orthogonale Aufteilung}
\begin{figure}
    \center
    \includegraphics{images/Laneorthogonal.pdf}
    \caption{Lane-orthogonale Aufteilung des Datenblocks}
    \label{fig:2_block_laneorthogonal}
\end{figure}
Spaltet man die Daten wie in Abbildung \ref{fig:2_block_laneorthogonal} gezeigt, sodass jeder Block jeweils 32 der insgesamt 64 Slices enthält,
so kann jedes Atom sehr einfach die Gamma-Funktion für die von ihm gehaltenen Teil berechnen. Einzig die Slices 31 und 63 müssen ausgetauscht werden,
da die Theta-Funktion für jeden Slice auch den benachbarten linken Slice benötigt. Die Berechnung der Rho-Funktion ist allerdings ein wenig komplizierter,
da jede Lane durch Rho unterschiedlich weit rotiert wird. Es muss also im Kommunikationsprotokoll für jede Lane extra festgelegt welche Bits genau in welchem
Takt ausgetauscht werden.

\subsubsection{2-Block Spalten-orthogonale Aufteilung}
\label{cha:iteration_1_optimierungen_spaltenorthogonal}
\begin{figure}
    \center
    \includegraphics{images/Spaltenorthogonal.pdf}
    \caption{Spalten-orthogonale Aufteilung des Datenblocks}
    \label{fig:2_block_spaltenorthogonal}
\end{figure}
Spaltet man die Daten entlang der Lanes wie in Abbildung \ref{fig:2_block_spaltenorthogonal}, so muss für die Gamma-Funktion jeder Slice
einmal zwischen den Atomen ausgetauscht werden. Die Berechnung kann allerdings weiterhin parallel erfolgen. Auch die Rho-Funktion kann
parallel berechnet werden und benötigt keinerlei Kommunikation.

\subsubsection{Anmerkung zu Zeilen-orthogonalen Mustern}
Die Gamma-Funktion benötigt ganze Slices für die Berechnung, wenn ein Slice in einem Schritt berechnet werden soll.
Daher bestehen für Zeilen-orthogonale Aufteilungsmuster exakt die gleichen Vor- und Nachteile wie für Spalten-orthogonale Muster.
Einzig für das Ergebnis ist ein Spalten-orthogonales Muster vorteilhaft, da das Endergebnis aus den ersten 4 Lanes besteht und nur dort alle 4 Lanes in einem Atom enthalten sind.

\subsubsection{4-Block Muster}
Für die Aufteilung in 4 Blöcke können so die vorherigen Muster mehrfach angewendet oder auch miteinander kombiniert werden.
Der Speicheraufwand pro Atom sinkt hier zwar auf etwa 25\% der gesammten Datenmenge, jedoch ist der zu erwartende Gewinn an LUTs nicht mehr so groß wie beim Schritt von 1 auf 2 Atome.
Zudem steigt der Kommunikationsaufwand deutlich an, was nicht nur eine erhöhte Ausführungszeit mit sich bringt, sondern auch wieder mehr Platz im Atom benötigt.
\newpage
\section{Zweiter Entwurf}

\subsection{Entwurfsziele}
Im zweiten Entwurf sollen die Ideen aus dem vorherigen Abschnitt möglichst effizient umgesetzt werden.
Das heißt der Beschleuniger wird in zwei Atome aufgeteilt, um die Datenmenge in einem Atom zu reduzieren
und statt der Standard-Rundenfunktion wird die erweiterte Rundenfunktion in den zwei Teilschritten Gamma und Rho implementiert.
Damit die Komponenten miteinander arbeiten können und die Atome Daten miteinander austauschen können, braucht es zusätzlich noch
einen Zustandsautomaten, der das Verhalten der Komponenten kontrolliert.
Die Atome sollen so klein sein wie möglich und dürfen dabei ruhig ein wenig die Ausführungszeit erhöhen.
Ziel des Entwurfs ist es die bisherigen Überlegungen zu testen und sicher zu stellen, dass sie die gewünschte Auswirkung zeigen.

\subsection{Aufbau}
[Schaubild]
[Wichtig: Erklärung Atom Index]

\subsection{Ablauf einer Berechnung}
\subsubsection{Dateneingabe}
Über den Datenbus werden die Atome mit den Eingabedaten versorgt. Die Eingabe wird entsprechend mit den bereits gespeicherten über ein XOR kombiniert.
Auf diese Weise können der Schwammkonstruktion entsprechend neue Datenblöcke direkt in den Atomen aufgenommen werden ohne,
dass das Ergebis der vorherigen Berechnung erst gelesen werden muss.
\subsubsection{Rho}
Für die Berechnung von Rho werden alle Lanes in einem Atom gleichzeitig in einem Takt wie in einem Schieberegister entsprechend rotiert.
\subsubsection{Gamma}
Die Berechnung von Gamma wird in mehrere Teilschritte aufgeteilt. Atom 0 ist dabei für die Berechnung der Slices 0 bis 31 zuständig und Atom 1
führt die Berechnung für die Slices 32 bis 63 durch. Damit ein neuer Slice berechnet werden kann, muss der vollständige Slice im Atom vorliegen.
Da jeweils 13 Bit schon im eigenen Datenspeicher vorhanden sind, müssen noch 12 Bits aus dem Speicher des anderen Atoms übertragen werden.
Nach der Berechnung muss das Ergebnis in beiden Atomen übernommen werden. Dafür müssen wieder 13 Bits pro Slice übertragen werden.
Um das Kommunikationsprotokoll so einfach wie möglich zu halten und die Ausführungszeit zu minimieren, wird die Berechnung ge-pipelined.
Der 64 Bit breite Voll-Duplex-Kanal zwischen den Atomen wird aufgeteilt in einen 32 Bit breiten Datenkanal und einen 32 Bit breiten Ergebniskanal.
Der genaue Berechnungsverlauf ist noch einmal im Schaubild [Bild Nr] erklärt.
[Schaubild]
\begin{enumerate}
\item Die Daten für Atom 1 werden aus dem Datenspeicher gelesen und an den Datenkanal angelegt.
\item Die Daten für Atom 1 befinden sich im Register des Datenkanals
\item Die Daten für Atom 1 sind am Atom eingetroffen. Gleichzeitig treffen auch die von Atom 1 gesendeten Daten an Atom 0 ein.
Die erhaltenen Daten werden mit den Daten aus dem Speicher von Atom 0 zu vollständigen Slices kombiniert und der Berechnungseinheit bereitgestellt.
\item Die Berechnungseinheit berechnet das Ergebnis und gibt es zurück.
\item Das Ergebnis wird im Datenspeicher übernommen und die Hälfte, die in Atom 1 gespeichert werden soll, wird am Ergebniskanal angelegt.
\item Das Erbebnis im Ergebnisbus befinden sich im Register des Ergebniskanals
\item Das Ergebnis trifft in Atom 1 ein. Gleichzeitig trifft auch das Ergebnis von Atom 1 in Atom 0 ein.
Das erhaltene Ergebnis wird im Datenspeicher übernommen.
\end{enumerate}
Die maximale Anzahl an Slices, die gleichzeitig in einem Atom berechnet werden kann, ergibt sich in diesem Fall aus der stark beschränkten Bandbreite
des Kommunikationskanals. In dem 32 Bit breiten Kanal können maximal zwei 12 Bit bzw 13 Bit Einträge in einem Takt übertragen werden.
Um den gesammten Blockteil zu berechnen, muss die oben aufgeführte Berechnungsabfolge also 16 Mal mit jeweils 2 Slices durchgeführt werden.
Da die Berechnung der 7 Schritte in einer Pipeline durchgeführt wird, beträgt die Berechnungsdauer nicht 16 * 7 = 112 Schritte,
sondern nur 7 + (16 - 1) = 22 Schritte.

\subsection{Bewertung}
Die Ausführungszeit für eine Iteration der modifizierten Rundenfunktion ist wie erwartet etwa um einen Faktor 30 langsamer als die Implementierung der ersten Iteration.
Dies ist wie bereits erklärt hauptsächlich der Aufteilung der Gamma-Funktion in 16 Teilschritte geschuldet, sowie der damit einhergehenden Verzögerung.
Anders jedoch als erwartet, ist die Größe der Atome durch das Aufteilen der Berechnung und des Datenspeichers nicht wie gewünscht gesunken.
Tatsächlich ist der Entwurf mit seinen etwa 4600 LUTs nochmal um gut 37\% größer. Dafür gibt es zwei wesentliche Gründe für den Zustandsautomaten sowie die Speicherkomplexität,
die in der Überlegung für das Design nicht bedacht wurden.
\subsubsection{Zustandsautomat}
Der Zustandsautomat besteht aus einem Iterator, der anhand des aktuellen Wertes die Steuersignale für die anderen Komponenten genertiert.
Entgegen der ursprünglichen Annahme, dass seine Größe aufgrund der Einfachheit der Aufgabe vernachlässigbar ist, nimmt er in diesem Design etwa 300 LUTs ein.
Auch wenn sich die konkrete Implementierung noch optimieren lässt, so ist klar geworden, dass die weitere Erhöhung der Berechnungskomplexität mit Bedacht durchgeführt werden muss,
da der Zustandsautomat dadurch nur noch größer wird.
\subsubsection{Schreib- und Lesemuster}
Im ersten Entwurf wird der Wert jedes Bits im Register entweder von der Eingabe oder von der Ausgabe der Rundenfunktion bestimmt.
Im zweiten Entwurf hingegen hängt dieser Wert ab von der Eingabe, des Ergebnisses der Rho-Rotation sowie einem Bit im Ergebnis-Kanal.
Welches Bit aus dem Ergebniskanal für ein Bit im Datenspeicher bestimmt ist, legt der Zustandsautomat und auch der Atom-Index fest.
Diese Auswahlschaltung sowie die Entscheidung, wann genau das Bit im Register überschrieben werden muss (im ersten Entwurf wurden einfach alle Bits in jedem Takt überschrieben, wenn das Kontrollsignal aktiv war),
benötigen schon mehr Platz als die Reduktion der Datenmenge einspart.
Analog ist auch das Lesen der Daten komplizierter geworden. Die Gamma-Funktion sowie der Datenkanal müssen anhand des Zustandsautomaten und des Atom-Indexes aus allen Bits nur ein par auswählen.
\subsubsection{Gamma-Funktion}
Die Gamma-Funktion übernimmt bis auf Rho alle Subfunktionen der Standard-Rundenfunktion. Da die Berechnung auf zwei Atome aufgeteilt ist und auch nicht alle Slices in einem Atom gleichzeitig berechnet werden,
ist die Komplexität der Gamma-Funktion sehr stark geschrumpft, sodass das aktuelle Design nur etwa 70 LUTs benötigt. Eine weitere Optimierung der Gamma-Funktion ist daher auch in weiteren Iterationen nicht mehr nötig.

\subsection{Optimierungsansätze}
Die starke Steigerung der Speicherkomplexität ist das Hauptproblem des Entwurfs und weitere Verbesserungen müssen hier ansetzen, um die Größe des Beschleunigers aus die erforderliche Größe reduzieren zu können.
Um die Speicherverwaltung vollständig aus dem Design zu entfernen, hatten wir die Nutzung der BRAM-Blöcke in den Überlegungen des ersten Entwurfs schon einmal erwähnt und uns letztendlich dagegen entschieden,
weil das Festlegen auf einen Tile-orientierten Speicher bedeutet, dass die Rho-Funktion, die eigentlich auf Lanes arbeitet, so implementiert werden muss, dass sie mit Tiles arbeiten kann.
\subsubsection{Transformation der Rho-Funktion}


\subsubsection{BRAM als Datenspeicher}

\subsubsection{Rho-Transformation}

Das Berechnen der Rho-Funktion auf Tile orientierten Daten kann mit Hilfe eines Schieberegisters als Puffer realisiert werden.
Schreibt man alle Rotationen, die größer als 32 Stellen sind, als Rechts-Rotationen um, so müssen in jedem Atom maximal 7 Links- und Rechts-Rotationen durchgeführt werden.
Hier bietet es sich wieder an, die Berechnung in zwei Schritte aufzuteilen. Im ersten Schritt werden die Links-Rotationen berechnet, während die anderen Lanes nicht verändert werden.
Im zweiten Schritt werden dann analog die Rechts-Rotationen durchgeführt. Um das Prinzip zu verdeutlichen, schauen wir uns zuerst die Verarbeitung einer einzelnen Lane an.
Für die Rotation um k Bits einer Lane l mit k < 32, kann die obere Hälfte von l rechts an l angefügt werden.
w = l[63 downto 0] || l[63 downto 32]
rotl(l, k) = w[95 - k downto 31 - k]

Interessant ist, dass diese Berechnung über ein 32 Bit Schieberegister realisiert werden kann.

b[31 downto 0] := l[63 downto 32]
for i in 0 to 63 loop
    r[i] := b[32 - k]
    b := (b >> 1) | (l[i] << 63)
end loop
return r

Mit diesem Vorgehen können die Rotationen berechnet werden, wobei nicht die ganze Lane, sondern immer nur die Hälfte in einem Puffer vorliegen muss.
Da die Berechnung der Rechts-Rotationen analog funktioniert, kann ein Puffer für beide Arten von Rotation verwendet werden.
Des weiteren lässt sich die schleife auch abrollen, womit mehrere Bits auf einmal verarbeitet werden können und natürlich können mit mehreren Puffern auch
mehrere Lanes gleichzeitig rotiert werden. Dabei unterscheiden sich die einzelnen Lanes nur im Index der Auswahl ihrer Ergebnisbits.
Für den konkreten Fall reichen insgesamt 7 Puffer aus. So können im ersten Schritt alle Links- und im zweiten Schritt alle Rechts-Rotationen berechnet werden.
[Schaubild]

\subsubsection{BRAM als Datenspeicher}
Mit dem neuen Ansatz für die Berechnung der Rho-Funktion kann auch der Datenspeicher in den BRAM verschoben werden, da das Problem der unterschiedlichen Ausrichtung der benötigten Eingabedaten behoben ist.
Ein BRAM Block unterstütz dabei bis zu zwei Lese- und Schreibports. Das ist essenziell für die Berechnung der Gamma-Funkion, da durch die Pipeline in jedem Takt sowohl Daten für die eigenen Berechnungen als
auch für die Berechnungen des anderen Atoms gelesen werden müssen und auch die Ergebnisse beider Atome gleichzeitig festgehalten werden.
Da die Gamma-Funktion immer zwei Slices gleichzeitig verarbeitet, bietet sich dieses Format auch für die Speicherstruktur an. So können von jedem Port immer zwei Tiles gleichzeitig adressiert werden.
Da jeder Atom-Contrainer über insgesammt 3 BRAM Blöcke verfügt, können die Ergebnisse der Gamma-Funktion auch in einem anderen BRAM-Block gespeichert werden.
Die Rho-Funktion kann tatsächlich quasi inplace in einem BRAM-Block berechnet werden, da der BRAM read-before-write unterstützt. Wird ein Tile k gelesen und liegt das Ergebnis n Takte später vor,
so kann es an der Stelle k + n gespeichert werden, nachdem im gleichen Takt der alte Slice mit dem Index k + n gelesen wurde.
Das bedeutet, dass beide Ports gleichzeitig Daten für die Berechnung bereitstellen können, sodass immer 4 Tiles gleichzeitig in den Puffer eingelesen werden können.
Auf diese Weise benötigt die Berechnung einer vollständigen Rotation somit theoretisch etwa 8 Takte zum Füllen des Puffers mit den Initialwerten und 16 Takte zum Lesen Schreiben aller Tiles zuzüglich
zu ein paar Verzögerungstakten aufgrund des BRAMs.

\subsubsection{Datenbus}
Da sowohl Gamma als auch Rho in der Zusammenarbeit mit dem BRAM wie es oben beschrieben ist, niemals auf zwei unterschiedlichen BRAM-Blöcken gleichzeitig schreiben oder gleichzeitig lesen,
können die Datenleitungen für die beiden Speicherblöcke zusammengelegt werden.
[Schaubild + Erklärung]


\newpage
\section{Finaler Entwurf}
\subsection{Entwurfsziele}
Mit den vorgestellten Überlegungen soll versucht werden das Design endlich auf die erforderliche Größe zu schrumpfen.
Die Ausführungszeit sollte dabei nicht mehr als bereits in den Überlegungen beschrieben.

\subsection{Aufbau}
[Schaubild + Erklärung]

\subsection{Funktion}
Da die Daten im BRAM nur Slice orientiert als Tiles gespeichert werden und die Eingabedaten aber als Lanes vorliegen, müssen sie erst konvertiert werden.
Diese Konvertierung findet während der vorherigen Berechnung in den B-Atomen statt.
Dabei lesen die B-Atome mehrfach den gesammten Datenblock und puffern immer einen Teil aller Lanes, bis sie sie als vollständige Tiles speichern können.
Dass dabei jede Lane mehrfach eingelesen werden muss, hat keinen Einfluss auf die Ausführunszeit des Beschleunigers, da die Berechnungszeit der modifizierten Rundenfunktion größer ist,
als die Berechnungszeit der Konvertierer und beide Systeme vollständig parallel arbeiten können.
Nachdem ein Block konvertiert wurde und die A-Atome bereit sind, werden die Tiles der Reihe nach über die Datenschnittstelle übertragen und in der Leseeinheit mit den Daten aus dem RES-Block
über ein XOR kombiniert und anschließend wieder in den RES-Block übernommen.
Für die Gamma-Berechnung werden die Slices über den Lesebus aus dem RES-Block gelesen an die Recheneinheit sowie an die Kommunikationseiheit übergeben.
Die Berechnung selbst funktioniert genau wie im zweiten Entwurf, tatsächlich wird die Recheneinheit aus dem vorherigen Entwurf wiederverwendet.
Die Ergebnisse werden anschließend von der Kommunikationseinheit und der Recheneinheit über den Schreib-Bus in den GAM-Block geschrieben.
Anschließend wird die Links-Rotation mit Hilfe des Rho-Puffers durchgeführt. Die Daten werden aus dem GAM-Block über den Datenbus an den Puffer übertragen.
Zuerst wird der Puffer mit den Initialdaten gefüllt, dann werden wie im Abschnitt [welcher Abschnitt ist denn das?] die Tiles nach und nach in den Puffer eingeschoben
und die Erbebnis-Tiles werden über den Schreib-Bus wieder in den GAM-Block übernommen.
Die Rechts-Rotation wird genau so durchgeführt, nur werden dieses Mal die Ergebnisse in den RES-Block geschrieben anstatt in den GAM-Block.
Hier wird auch ein Vorteil des Bus-Designs deutlich: Da beide Blöcke ihre Daten über den Bus empfangen, braucht für die Rechts-Rotation nur das
Write-Enable-Signal verändert werden und die Ergebnisse müssen nicht auf eine andere Eingabe umgeleitet werden.
Die Rho und Gamma Funktionen werden so oft berechnet, bis der Block gemäß der KECCAK-P Funktion verarbeitet wurde.
Danach kann entweder über das Kontrollsignal bestimmt werden, ob der nächste Block eingelesen werden soll, oder das Ergenis ausgegeben werden soll.
Da das Ergebnis aus den ersten 4 Lanes der gespeicherten Daten besteht, müssen die Tiles wieder zu Lanes konvertiert werden.
Dabei ist keine Kommunikation zwischen den Atomen notwendig, da der Datenblock Spalten-orthogonal aufgeteilt wurde und so alle benötigten Lanes sich in Atom 0 befinden.
Es werden über den Datenbus alle 64 Tiles nacheinander ausgelesen und die jeweils 4 ersten Bits in einem extra Puffer gespeichert.
Liegt das gesammte Ergebnis im Puffer, so wird es über das Interface nach außen ausgegeben.

\subsection{Bewertung}

\subsection{Weitere Optimierungsansätze}

\subsubsection{Reduktion auf 1 Atom}
Die Berechnung wurde ursprünglich auf zwei Atome aufgeteilt, damit die Datenmenge reduziert werden kann, die ein Atom halten muss.
Durch die Einführung des BRAM fällt diese Optimierung weg, da der BRAM mehr als genug Platz bereitstellt.

\subsubsection{Erweiterung des Speicherinterfaces auf mehr Tiles}
Die Aufteilung des Speichers in zum Beispiel 4 Tiles erlaubt es eine größere Menge an Tiles gleichzeitig für die Berechnung von Rho bereit zu stellen.
Da in einem zwei Atom Ansatz die Gamma-Funktion durch die Bandbreite des Datenkanals begrenzt ist, würde dieser Teil keine Beschleunigung erfahren.
In einem Ein-Atom-Ansatz hingegen würden allerdings auch hier noch mehr Slices gleichzeitig berechnet werden können.

\subsubsection{Erweiterung des Rho-Puffers}
Durch 1 Atom Ansatz müssen alle Lanes in einem Atom berechnet werden, kann der Puffer erweitert werden oder braucht das zu viel Platz?
Müssen mehr als 2 Rotationen durchgeführt werden?

\subsubsection{Auslagerung der Rho-Berechnung}
Auch wenn alle Daten in einem Atom gespeichert werden, könnte es sinnvoll sein, die Berechnung von Rho auf verschiedene Atome zu verteilen.
[Schaubild]. So können beim Filtern der Lanes die restlichen Lanes an ein anderes Atom übertragen werden, das die Rotation durchführt.
So kann die Größenbeschränkung des Puffers umgangen werden. Wird auch dieser Prozess durch eine Pipeline realisiert, beträgt die maximale Bandbreite hierfür
max k s.t. k * 7 <= 32
=> k = 4 Slices. Wird der Puffer im Hauptatom noch ein wenig erweitert, sodass maximal 6 Lanes gleichzeitig ausgelagert werden müssen, so steigt die Bandbreite sogar
auf 5 Slices. Da allerdings immer zwei Ports für die Zusammensetzung der Daten verantwortlich sind, müssten immer 6 Slices gleichzeitig gelesen werden und die
Restlichen in einem weiteren Puffer zwischengehalten werden was die Komplexität weiter erhöht. Außerdem ist ein solcher Ansatz nicht mehr symmetisch,
heißt es kann nicht der gleiche Beschleuniger auf beide Atome geladen werden.


\chapter{Ergebnisse}
Die drei vorgestellten Designs für den Beschleuniger bauen aufeinander auf mit dem Ziel immer mehr Leistung für weniger Platzbedarf einzutauschen.
In diesem Kapitel werden alle Designs miteinander verglichen was sowohl ihre Größe angeht, als auch ihr Leistungspotential.
Das konkrete Ziel dieser Arbeit bestand darin einen Beschleuniger zu entwerfen, der speziell die Anforderungen der i-Core-Architektur erfüllt.
Daher wird in diesem Kapitel hauptsächlich der i-Core als Vergleich herangezogen für die Ausführung von Software-Implementierungen.
Außerdem werden wir uns bei der Untersuchung der Ausführungszeiten auf die KECCAK-p-Funktion beschränken,
da diese den Hauptaufwand darstellt und gerade bei größeren Datenmengen Schritte wie die Initialisierung und das Auslesen des Hashs am Ende vernachlässigbar sind.
Bei Vergleichen zwischen Beschleunigern mit Software-Berechnungen dient immer die Implementierung aus Anhang \ref{cha:anhang1} als Bezug.

\section{Syntheseergebnisse}
\begin{table}
    \centering
    \begin{tabular}{lrrrrr}
    & A-Atome & B-Atome & LUTs & FFs & BRAM Bänke \\
    \hline
    1. Entwurf & 1 & 0 & 3314 & 1845 & 0 \\
    2. Entwurf & 2 & 0 & 4643 & 1865 & 0 \\
    3. Entwurf & 2 & 2 & 1205 & 628 & 2
    \end{tabular}
    \label{tab:synth_ergebniss}
    \caption{Interessante Größen der synthetisierten Designs}
    \small
    Sämtliche Größen beziehen sich auf die A-Atome
\end{table}
In Tabelle \ref{tab:synth_ergebniss} sind die Größen der verschiedenen Beschleunigeriterationen noch einmal zusammengefasst.
Während die ersten beiden Iterationen alle Funktionalitäten in die A-Atomen integrieren, die die tatsächliche Berechnung der KECCAK-p-Funktion durchführen,
benötigt die dritte Iteration noch zwei zusätzliche B-Atome, in denen die Eingabedaten so umgeordnet werden, dass sie in den BRAM der A-Atome geschrieben werden können.
Da in dem dritten Entwurf die Obergrenze von 1600 LUTs für die A-Atome noch nicht erreicht ist, kann diese Funktionalität auch in die Beschleuniger integriert werden.
Je weniger Atome für den Beschleuniger benötigt werden, desto schneller kann der Beschleuniger geladen werden.
Performance-technisch ist es jedoch sinnvoller diese Funktionen zu trennen. Da die A-Atome bei der Berechnung sowohl den BRAM als auch ihre Kommunikationsschnittstelle
voll auslasten, kann die Aufgabe der B-Atome nicht gleichzeitig in den A-Atomen durchgeführt werden. Daher müsste für die Zusammenführung der Atome die Einlese-Phase deutlich verlängert werden.
Für das Verarbeiten großer Datenmengen ist es daher sinnvoll die leicht erhöhte Konfigurationszeit des Beschleunigers in Kauf zu nehmen.

\section{Ausführungszeit}
Zur Bestimmung der Ausführungszeit des Beschleunigers verwenden wir hier die Simulation. Das hat den Vorteil,
dass auch die vorherigen Entwürfe mit verglichen werden können, die zu groß für den i-Core sind.
Da alle Entwürfe statisch sind in ihrer Ausführungszeit, die Abarbeitung eines Blocks also immer die gleiche Zeit benötigt,
liefert diese Herangehensweise genauere Werte als die Zeitmessung es auf dem i-Core tun würde.
Da aller Entwürfe den Betrieb auf dem 50MHz Takt des i-Cores zulassen, können die Taktzahlen direkt miteinander verglichen werden, siehe Tabelle \ref{tab:zeiten_alle_iterationen}.
\begin{table}
    \centering
    \begin{tabular}{lrrr}
        & Berechnungsdauer & Einlesedauer & Gesamtdauer\\
        \hline
        1. Entwurf & 24 Takte & 34 Takte & 58 Takte \\
        2. Entwurf & 509 Takte & 34 Takte & 543 Takte \\
        3. Entwurf & 1968 Takte & 36 Takte & 2004 Takte
    \end{tabular}
    \label{tab:zeiten_alle_iterationen}
    \caption{Ausführungszeiten der verschiedenen Beschleuniger-Entwürfe}
    \small Alle Zeitangaben beziehen sich auf die Abarbeitung eines Datenblocks. Initialisierung und Auslesen des Hashs sind nicht einberechnet.
\end{table}
Die Einlesedauer der Iterationen unterscheidet sich dabei fast gar nicht. In den ersten beiden Entwürfen liegt jede der 17 Lanes aus den Eingabedaten
nacheinander jeweils einen Kontrollschritt, also zwei Takte, an beiden Atomen an und die Atome wählen jeweils die Lanes aus, die sie benötigen
(im ersten Entwurf liest das einzige Atom alle Lanes ein).
Im dritten Entwurf hingegen lesen beide A-Atome gleichzeitig verschiedene Slices, die ihnen von den B-Atomen bereitgestellt werden.
Da in jedem Kontrollschritt jedoch nur jeweils vier Tiles zu je 13 Bits übertragen werden anstatt einer ganzen Lane, werden 16 statt der vorherigen 13 Kontrollschritte benötigt.
Diese Effekte heben sich gegenseitig auf. Eine weitere Optimierung der Einlesezeiten ist auch nicht weiter nötig. Im ersten Entwurf ist sie nicht möglich,
da die Kommunikationsschnittstelle vollständig ausgelastet ist und in den beiden letzten Entwürfen beträgt das Einlesen einen zu kleinen Teil an der gesamten Berechnung.

Die tatsächliche Berechnung hingegen nimmt mit jedem Entwurf deutlich mehr Zeit in Anspruch. Das ist alleine der Sequenzialisierung geschuldet.
Statt wie vorher die gesamte Rundenfunktion in einem Takt abzuarbeiten, werden im zweiten Entwurf Rho in einem Schritt und Gamma/Theta in 16 Schritten für jeweils 2 Slices pro Atom berechnet.
Hinzu kommen noch ein paar Takte durch die Verzögerung zwischen den Atomen, sodass die Berechnung der erweiterten Rundenfunktion im zweiten Entwurf insgesamt 21 Takte benötigt.
Der dritte Entwurf verwendet die gleiche Vorgehensweise für die Berechnung von Gamma/Theta, braucht allerdings drei Takte länger durch die Verzögerung des BRAM
und berechnet außerdem auch Rho sequenziell in 57 Takten. In Tabelle \ref{tab:zeiten_iteration_3} sind die Ausführungszeiten
für die verschiedenen Teilfunktionen im dritten Entwurf noch einmal zusammengefasst.
\begin{table}
    \centering
    \begin{tabular}{lrrr}
    Funktion & Zeit (in Takten) & Iterationen in KECCAK-p & Anteil an KECCAK-p \\
    \hline
    Initialisierung & 18 & - & - \\
    Einlesen & 36 & 1 & 1,8\% \\
    Theta & 24 & 1 & 1,2\% \\
    Gamma & 24 & 24 & 28,74\% \\
    Rho & 57 & 24 & 68,26\% \\
    KECCAK-p & 2004 & 1 & 100\% \\
    Auslesen & 27 & - & -
    \end{tabular}
    \label{tab:zeiten_iteration_3}
    \caption{Ausführungszeiten des finalen Beschleunigers}
\end{table}
Die KECCAK-p-Funktion, für die wir uns eigentlich bei der Berechnung von SHA-3 interessieren, setzt sich zusammen aus dem Einlesen des Datenblocks,
einer Berechnung von Theta, sowie 24 Berechnungen von Rho und Gamma.

\section{Weitere Optimierungsansätze}
Der dritte Entwurf des Beschleunigers erfüllt zwar alle geforderten Voraussetzungen, trotzdem lassen sich noch weitere Verbesserungen vornehmen, um die Leistungsfähigkeit zu erhöhen.
Daher wollen wir uns hier noch ein paar dieser möglichen Optimierungen anschauen.

\subsection{Auslagerung der Ergebniskonvertierung}
Das Ergebnismodul nimmt unnötig viel Platz im Atom ein und ist eigentlich nur deshalb in den A-Atomen enthalten, weil der Platz nicht weiter benötigt wird.
Es lässt sich aber auch in die B-Atome verschieben, die auch die Konvertierung für die Eingabe übernehmen.
So kann noch ein wenig mehr Platz für andere Optimierungen geschaffen werden, die die Berechnungsgeschwindigkeit weiter verbessern.

\subsection{Reduktion auf ein A-Atom}
Die Berechnung wurde ursprünglich auf zwei Atome aufgeteilt, damit die Datenmenge reduziert werden kann, die ein Atom halten muss.
Durch die Verwendung des BRAM für den Datenspeicher fällt dieser Aufwand weg, da der BRAM mehr als genug Platz bereitstellt.
Ein Beschleuniger, der nur ein A-Atom verwendet, bietet vor allem den Vorteil, dass für die Konvertierung der Eingabedaten auch nur ein B-Atom benötigt wird.
Der gesamte Beschleuniger benötigt also nur zwei der fünf Atome. Abhängig vom Anwendungsfall können so auch mehrere kleine Beschleuniger nebeneinander existieren,
ohne dass die Atom-Container zwischendurch rekonfiguriert werden müssen.
Außerdem ist die Berechnung selbst nicht mehr durch die Kommunikationsschnittstelle zwischen den Atomen beschränkt, wodurch weitere Optimierungen möglich werden.
Die Reduktion auf ein A-Atom alleine bringt jedoch keine direkte Leistungsverbesserung. Im Gegenteil, da aktuell beide A-Atome parallel sowohl Rho als auch Gamma berechnen,
würde die Reduktion alleine die Berechnungszeit etwa verdoppeln. Auch das Einlesen der Datenblöcke dauert länger, da der gesamte Block in ein einziges Atom übertragen werden muss,
was wiederum durch die Kommunikationsschnittstelle beschränkt ist. 

\subsection{Erweiterung der BRAM-Schnittstelle}
Aktuell können an einem Port des BRAM immer jeweils zwei Tiles gleichzeitig gelesen und geschrieben werden.
Die Erweiterung der Speicherschnittstelle auf zum Beispiel 4 Tiles erlaubt es,
eine größere Menge an Tiles gleichzeitig für die Berechnung von Rho bereit zu stellen.
Die Rho-Funktion könnte somit doppelt so schnell berechnet werden. Da die Berechnung von Rho mit 57 Takten
etwa 70\% der 81 Takte für die Berechnung der erweiterten Rundenfunktion beansprucht (siehe \ref{tab:zeiten_iteration_3}),
ist nochmal mit einer weiteren Beschleunigung von $S_{WideBRAM} = \frac{81\ Takte}{81\ Takte - (57\ Takte / 2)} \thickapprox 1,54$ zu rechnen.
Die restlichen 30\% der Berechnungszeit für die erweiterte Rundenfunktion werden für die Gamma-Funktion verwendet.
Diese kann durch die Erweiterung der Speicherschnittstelle nicht weiter beschleunigt werden,
da sie die Kommunikationsschnittstelle zwischen den Atomen bereits den limitierende Faktor darstellt.
Unklar ist noch, ob der vorhandene Platz für diese Optimierung ausreicht.

\subsection{Erweiterung des Rho-Puffers}
Möchte man wie oben beschrieben den Beschleuniger auf ein Atom reduzieren, kann auch der Rho-Puffer,
in dem die Bit-Rotationen blockweise auf mehreren Lanes gleichzeitig durchgeführt werden, erweitert werden.
Im aktuellen Entwurf ist das nicht mehr möglich, da genau eine Links- und eine Rechtsrotation auf
etwa jeweils der Hälfte der in den Atomen gespeicherten Lanes durchgeführt werden.
Links- und Rechtsrotationen lassen sich nicht zusammenlegen, was damit zusammenhängt in welche
Richtung die Lanes aus dem Speicher gelesen und geschrieben werden.
Findet jedoch die gesamte Berechnung nur in einem Atom statt, sind für die vollständige Abarbeitung
zwei Links- und zwei Rechts-Rotationen notwendig. Erweitert man den Puffer, sodass jeweils die beiden Links- und die beiden Rechts-Rotationen
zusammengelegt werden können, kann dadurch verhindert werden, dass die Berechnung der Rho-Funktion länger dauert.

\subsection{Auslagerung der Rho-Berechnung}
Es ist zu erwarten, dass die Erweiterung des Rho-Puffers viel zusätzlichen Platz in Anspruch nehmen wird.
Sollte das Atom dadurch die maximale Größe überschreiten, kann eventuell ein Teil der Berechnung von Rho in ein anderes Atom wieder ausgelagert werden.
Aktuell werden in jedem Takt jeweils 4 Bits aus maximal 7 Lanes, also maximal 28 Bit in den Puffer geschrieben und gleichzeitig gelesen.
Dieser Datenverkehr ist über die Kommunikationsschnittstelle durchaus gleichzeitig in beide Richtungen möglich, er kann sogar noch verdoppelt werden,
wenn man auch die BRAM-Schnittstelle wie oben beschrieben erweitert. Jenachdem wie groß diese Konstruktion wird, kann sie auch in das B-Atom integriert werden,
sodass der Beschleuniger trotzdem nur aus zwei Atomen besteht. Das B-Atom würde dann während der Berechnung von Rho mit dem A-Atom zusammenarbeiten und
zwischendurch den nächsten Datenblock einlesen.

\subsection{Erhöhung der Berechnungsfrequenz}
Der Beschleuniger selbst erlaubt eine Taktfrequenz von 200MHz, was in diesem Fall das Maximum für das verwendete FPGA ist. Der i-Core selbst
läuft jedoch nur mit einer Frequenz von 50MHz. Da der Beschleuniger seinen Takt mit dem i-Core teilt, damit die Synchronisierung am einfachsten ist,
läuft er jedoch sehr viel langsamer, als er eigentlich könnte. Verwendet man einen Takt von 200MHz
für die Berechnung der Rho-Funktion, die vollständig in den Atom erfolgt und keinerlei Kommunikation benötigt, erhält man eine Beschleunigung von
$S_{200MHz} = \frac{81\ Takte / 50MHz}{(81\ Takte - 57\ Takte) / 50MHz + 57\ Takte / 200MHz} \thickapprox 2,12$.
Für die Herkunft der genauen Taktzahlen siehe Tabelle \ref{tab:zeiten_iteration_3}. Auch hier kann wieder nur die Rho-Funktion beschleunigt werden,
da die Kommunikation zwischen den Atomen durch den i-core wieder durch den 50MHz-Takt beschränkt ist.

\section{Gemessene Beschleunigung}
Wie bereits erwähnt, sind die entworfenen Beschleuniger strikt deterministisch in der Hinsicht, dass die Dauer der Berechnung immer dieselbe Zeit benötigt.
Daher kann aus den vom Beschleuniger benötigten Takten sowie der Taktfrequenz direkt die Ausführungszeit des Beschleunigers bestimmt werden.
Bei der Software-Berechnung ist das nicht so einfach möglich, da zum Beispiel Speicherzugriffe aufgrund der Speicherstruktur unterschiedliche Ausführungszeiten benötigen.
So kann die Anzahl der auftretenden Cache Misses einen großen Einfluss auf die Berechnungsdauer haben. Auch ist nicht klar, ob wirklich jede Operation des Befehlssatzes
in genau einem Takt ausgeführt werden kann. Durch Datenabhängigkeiten kann es notwendig sein, dass die weitere Abarbeitung auf Ergebnisse vorheriger Instruktionen,
die sich noch in der Pipeline befinden, warten muss. Die beste Möglichkeit zur Bestimmung der Laufzeit der Software-Berechnung ist daher die tatsächliche Zeitmessung.
Diese Messung ist natürlich etwas ungenau, weshalb in 5 Durchläufen jeweils 10.000 Iterationen der KECCAK-p-Funktion durchgeführt werden, woraus dann die mittlere Berechnungsdauer bestimmt wird.
Die Ergebnisse dieser Messungen sowie der berechnete Durchschnitt $\tilde{t}$ sind in Tabelle \ref{tab:software_zeitmessung} aufgeführt.
\begin{table}
    \centering
    \begin{tabular}{lrrrrrr}
        Messung & 1 & 2 & 3 & 4 & 5 & Durchschnitt ($\tilde{t}$)\\
        \hline
        Zeit (s) & 14,505455 & 14,505454 & 14,505463 & 14,505464 & 14,505452 & 14,5054576
    \end{tabular}
    \label{tab:software_zeitmessung}
    \caption{Ausführungszeit der Software-Berechnung}
\end{table}
Aus dieser Messung ergibt sich für eine einzige Berechnung der KECCAK-p-Funktion eine Zeit von
\begin{align*}
    t_{SW} & = \tilde{t}/10000 = 14,5054576 s / 10000 = 1450,54576 \mu s
\end{align*}
Die Ausführungszeit der KECCAK-p-Funktion auf dem finalen Beschleuniger und der daraus resultierende Speedup kann
mit Hilfe der Tabelle \ref{tab:zeiten_iteration_3} aus dem 50MHz-Takt des i-Core wie folgt exakt berechnet werden:
\begin{align*}
    t_{ACC} & = \frac{2004\ Takte}{50 * 10^6Hz} = 0,00004008s = 40,08 \mu s \\
    S_{ACC} & = \frac{t_{SW}}{t_{ACC}}\frac{1450,54576 \mu s}{40,08 \mu s} \thickapprox \mathbf{36,2}
\end{align*}

\section{Theoretische Beschleunigung}
Neben der tatsächlichen Ausführungszeit auf dem i-Core wollen wir uns noch eine andere Metrik anschauen, um den Beschleuniger zu bewerten.
Dazu betrachten wir die Anzahl an elementaren Operationen, die die Software auf einem virtuellen System benötigt, um die KECCAK-p-Funktion zu berechnen.
Als elementare Operationen zählen dabei Instruktionen, von denen man erwarten kann, dass sie von jedem modernen Prozessor in jeweils einem Takt berechnet werden können.
Als solche Instruktionen zählen in diesem Fall XODER, UND, ODER, NEGATION (NEG), das Kopieren, eine Rotation um ein Bit nach links (ROL) sowie ein Bitshift variabler Länge.
Weiter unterschieden werden muss allerdings zwischen 32-Bit- und 64-Bit-Operationen. In der Tabelle \ref{tab:software_instruktionen} ist
die Anzahl der jeweils benötigten 64-Bit-Operationen aufgeführt, die für die einzelnen Teilfunktionen von KECCAK-p benötigt werden, die sich aus insgesamt 24 Runden zusammensetzt.
Die Anzahl an benötigten 32-Bit-Operationen ist dann doppelt so groß. Eine 64-Bit-Rotation variabler Länge, wie sie von Rho verwendet wird,
ist dabei, wie auch in der Implementierung im Anhang \ref{cha:anhang1} beschrieben, aus zwei Shifts und einem ODER zusammengesetzt.
Außerdem werden sämtliche Operationen, die nicht direkt zur die für die Ergebnisberechnung dienen,
wie zum Beispiel Zählvariablen oder Indexberechnungen für Felder, ignoriert.
\begin{table}
    \centering
    \begin{tabular}{lrrrrrrrr}
        & XODER & UND & ODER & NEG & Kopie & ROL & Shift & Gesamt \\
        \hline
        Theta & 50 & 0 & 0 & 0 & 0 & 5 & 0 & 55 \\
        Rho-Pi & 0 & 0 & 24 & 0 & 49 & 0 & 48 & 121 \\
        Chi & 25 & 25 & 0 & 25 & 25 & 0 & 0 & 100 \\
        Iota & 1 & 0 & 0 & 0 & 0 & 0 & 0 & 1 \\
        Rnd & 76 & 25 & 24 & 25 & 74 & 5 & 48 & 277\\
        KECCAK-p & 1824 & 600 & 576 & 600 & 1776 & 120 & 1152 & 6648
    \end{tabular}
    \label{tab:software_instruktionen}
    \caption{Instruktionen der Software-Funktionen}
\end{table}
Da wir gefordert haben, dass jede dieser Operationen in jeweils einem Takt berechnet können werden soll,
ergibt sich aus der Anzahl der Operationen auch gleich die Anzahl der benötigten Takte.
Auf diese Weise können wir wieder einen Speedup für den Beschleuniger berechnen, indem wir die benötigten Operationen
der Software mit den benötigten Takten des Beschleunigers vergleichen. Hier können auch alle Entwürfe einbezogen werden,
da wir für unsere virtuelle Umgebung keine Begrenzungen für die Größe der Beschleuniger vorgegeben haben und somit der Vergleich auch sinnvoll ist.
In Tabelle \ref{tab:theoretischer_speedup} ist sowohl der theoretische Speedup sowohl bezüglich 32-Bit- als auch 64-Bit-Operationen aufgeführt.
\begin{table}
    \centering
    \begin{tabular}{lrr}
        & 32-Bit-Speedup & 64-Bit-Speedup \\
        \hline
        1. Entwurf & 229,24 & 114,62 \\
        2. Entwurf & 24,49 & 12,25 \\
        3. Entwurf & \textbf{6,63} & 3,32
    \end{tabular}
    \label{tab:theoretischer_speedup}
    \caption{Theoretischer Speedup der verschiedenen Entwürfe}
\end{table}
Dieser theoretische Speedup beschreibt den Faktor, wie viel der tatsächlichen Berechnung der Beschleuniger in jedem Schritt mehr erledigt,
als die Software es im Optimalfall tut. Vergleicht man den theoretischen Speedup mit dem gemessenen Speedup, so stellt man fest,
dass der gemessene Speedup des finalen Entwurfs mit 36,2 um einen Faktor 5,46 größer ist als der theoretische Speedup von 6,63
(wir verwenden hier den 32-Bit-Speedup, da der i-Core eine 32-Bit-Plattform ist). Dieser Faktor beschreibt die gewonnene Geschwindigkeit,
die dadurch entsteht, dass der Beschleuniger keine Kontrollstrukturen wie Schleifen, Speicherzugriffszeiten, oder mehrere Takte für eine Instruktion benötigt.


\chapter{Fazit}


In dieser Arbeit wurde gezeigt, wie ein Hardwarebeschleuniger für eine rekonfigurierbare Prozessorplattform
wie den i-Core implementiert werden kann, die die starken Anforderungen, besonders was die maximale Größe des Beschleunigers betrifft, erfüllt.
Dazu wurden ausgehend von einem monolithischen Design mit sehr hoher Berechnungsgeschwindigkeit
nach und nach kleine Änderungen vorgenommen, worduch am Ende ein modulares System entstand, das zwar eine geringere
Berechnungsgeschwindigkeit besitzt, dafür aber die Anforderungen der Plattform erfüllt.



Für die vorgenommenen Verbesserungen haben wir verschiedene Ansätze diskutiert und damit die weiteren
Entwurfsentscheidungen begründet. Besonders hervorzuheben sind hier die Berechnungseinheiten für $\rho$ und $\gamma$,
von denen jeweils eine bereits ausreicht, um die gesamte erweiterte Rundenfunktion zu berechnen, die aber auch
je nach verfügbarem Platz mehrfach implementiert werden können, um eine weitere Beschleunigung zu erzielen.


Der konkret für den i-Core entwickelte Beschleuniger besteht aus insgesamt vier Atomen,
nutzt jedoch den Platz der den einzelnen Atomen zur Verfügung steht, nicht vollständig aus.
Daher wurden noch weitere vielversprechende Verbesserungsmöglichkeiten für die Zukunft vorgestellt,
die den noch verfügbaren Platz verwenden, um die Berechnungsgeschwindigkeit weiter zu erhöhen.
Als Beispiel sei da die Erweiterung der Speicherschnittstelle genannt, wodurch die Berechnungszeit der
$\rho$-Funktion, die etwa 70\% der gesamten Berechnungszeit beansprucht, nochmal halbiert werden kann.
Bei einem Vergleich des aktuellen Beschleuniger mit einer Software-Implementierung ließ sich feststellen,
dass der entworfene Beschleuniger die KECCAK-p-Funktion, welche bei SHA-3 den alleinigen Rechenaufwand darstellt,
um einen Faktor 36 schneller berechnet, als es die Software auf dem i-Core tut. Über eine genauere Untersuchung
der Berechnungsabläufe in der Software ist dabei klar geworden, dass etwa ein Faktor 6 der gemessenen Beschleunigung
durch die veränderte Berechnungsmethode entsteht und ein weiterer Faktor 6 durch Einflüsse wie Kontrollfluss-Strukturen
in der Software und Speicherzugriffe mit langen Zugriffszeiten durch zum Beispiel Cache Misses, entsteht.

\chapter{Ähnliche Arbeiten}


- Buehner 2022: Implementierung der inferenz für convolutional neural networks
- Kriebel 2009: Entwicklung von Beschleunigeratomen für SHA-1
- Riedlberger 2015: Implementierung der SHA-1 Spezialinstruktion
- Pöppl 2018: Modelling Shallow Water Waves

\chapter{Glossar}
\section{Begriffe}
\begin{tabularx}{\linewidth}{@{}>{\bfseries}l@{\hspace{.5em}}X@{}}
	Atom: & Beschleuniger, oder Teil eines Beschleunigers, der in einen Atom-Container geladen werden kann \\
	Atom-Container: & Programmierbarer Logikblock in einem rekonfigurierbaren Prozessor \\
	Atom-Index: & Laufzeitparameter des Beschleunigers, der die genaue Aufgabe des Atoms bestimmt \\
	Lane: & Ein eindimensionaler 64-Bit langer Ausschnitt aus einem State Array \\
	Slice: & Ein ein Bit breiter vertikaler Ausschnitt aus allen Lanes, insgesamt 25 Bits groß \\
	State Array: & Die dreidimensionale Darstellung des internen Zustandsvektors der KECCAK-p-Funktion \\
	Schwammkonstruktion: & Teil des SHA-3-Algorithmus, siehe \ref{cha:sha3} \\
	Tile: & Die 13 oberen oder 13 unteren Bits eines Slices (es gibt ein Bit, das in beiden enthalten ist)
\end{tabularx}
\newpage
\section{Abkürzungen}
\begin{tabularx}{\linewidth}{@{}>{\bfseries}l@{\hspace{.5em}}X@{}}
	AC: & Atom Conainer \\
	AGU: & Address Generation Unit \\
	BPP: & Bounded Error Probabilistic Polynomial Time \\
	BQP: & Bounded Error Quantum Polynomial Time \\
	BRAM: & Block RAM \\
	CLB: & Configurable Logic Block \\
	DSP: & Digital Signal Processor \\
	FPGA: & Field Programmable Gate Array \\
	IoT: & Internet of Things \\
	LSU: & Load-Store-Unit \\
	LUT: & Lookup Table \\
	SHA: & Secure Hash Algorithm \\
	SI: & Spezialinstruktion \\
	TLM: & Tile Local Memory \\
	VLCW: & Very Long Control Word
\end{tabularx}

\chapter{Symbolverzeichnis}
\input{chapters/symbolverzeichnis.tex}

\chapter{Anhang 1: Softwareimplementierung}
\label{cha:anhang1}
Für Vergleiche zwischen Software-Algorithmen und Hardware-Beschleunigern ist es wichtig, eine möglichst effiziente Software-Implementierung zu verwenden,
um ein aussagekräftigen Ergebnis zu erhalten. Daher wird für alle Vergleiche in Kapitel \ref{cha:ergebnisse} eine fremde Implementierung verwendet,
die viele Optimierungen enthält. Sie wird vom Github-Nutzer \textit{brainhub} \cite{sha3-impl} unter MIT-Lizenz zur Verfühgung gestellt (\ref{fig:keccak_impl_license}).
Im Folgenden sind die interessanten Segmente der KECCAK-p-Funktion aufgelistet, wobei einige Symbole zur besseren Lesbarkeit umbenannt wurden.

\begin{figure}
\lstset{language=C}
\begin{lstlisting}[label={lst:keccak_impl_license}]
MIT License

Copyright (c) 2020 brainhub

Permission is hereby granted, free of charge, to any
person obtaining a copy of this software and associated
documentation files (the "Software"), to deal in the
Software without restriction, including without limitation
the rights to use, copy, modify, merge, publish,
distribute, sublicense, and/or sell copies of the Software,
and to permit persons to whom the Software is furnished to
do so, subject to the following conditions:

The above copyright notice and this permission notice shall
be included in all copies or substantial portions of the
Software.

THE SOFTWARE IS PROVIDED "AS IS", WITHOUT WARRANTY OF ANY
KIND, EXPRESS OR IMPLIED, INCLUDING BUT NOT LIMITED TO THE
WARRANTIES OF MERCHANTABILITY, FITNESS FOR A PARTICULAR
PURPOSE AND NONINFRINGEMENT. IN NO EVENT SHALL THE AUTHORS
OR COPYRIGHT HOLDERS BE LIABLE FOR ANY CLAIM, DAMAGES OR
OTHER LIABILITY, WHETHER IN AN ACTION OF CONTRACT, TORT OR
OTHERWISE, ARISING FROM, OUT OF OR IN CONNECTION WITH THE
SOFTWARE OR THE USE OR OTHER DEALINGS IN THE SOFTWARE.
\end{lstlisting}
\caption{Lizenz der verwendeten SHA-3-Implementierung}
\label{fig:keccak_impl_license}
\end{figure}

\begin{figure}
\lstset{language=C}
\begin{lstlisting}[label={lst:keccak_impl_utils}]
static const uint64_t keccak_rount_constants[24] = {
    0x0000000000000001ULL, 0x0000000000008082ULL,
    0x800000000000808aULL, 0x8000000080008000ULL,
    0x000000000000808bULL, 0x0000000080000001ULL,
    0x8000000080008081ULL, 0x8000000000008009ULL,
    0x000000000000008aULL, 0x0000000000000088ULL,
    0x0000000080008009ULL, 0x000000008000000aULL,
    0x000000008000808bULL, 0x800000000000008bULL,
    0x8000000000008089ULL, 0x8000000000008003ULL,
    0x8000000000008002ULL, 0x8000000000000080ULL,
    0x000000000000800aULL, 0x800000008000000aULL,
    0x8000000080008081ULL, 0x8000000000008080ULL,
    0x0000000080000001ULL, 0x8000000080008008ULL
};

static const unsigned keccak_rho_distances[24] = {
    1, 3, 6, 10, 15, 21, 28, 36, 45, 55, 2, 14,
	27, 41, 56, 8, 25, 43, 62, 18, 39, 61, 20, 44
};

static const unsigned keccakf_pi_indices[24] = {
    10, 7, 11, 17, 18, 3, 5, 16, 8, 21, 24, 4, 15,
	23, 19, 13, 12, 2, 20, 14, 22, 9, 6, 1
};

#define ROTL(x, y) \
	(((x) << (y)) | ((x) >> ((sizeof(uint64_t)*8) - (y))))
#endif
\end{lstlisting}
\caption{Konstanten und Hilfsfunktionen der KECCAK-p-Implementierung}
\label{fig:keccak_impl_utils}
\end{figure}


\begin{figure}
\lstset{language=C}
\begin{lstlisting}[label={lst:keccak_impl}]
static void keccak_p(uint64_t stateArray[25]) {
  int i, j, round;
  uint64_t temp, buffer[5];
  #define KECCAK_ROUNDS 24

  for(round = 0; round < KECCAK_ROUNDS; round++) {

    /* Theta */
    for(i = 0; i < 5; i++)
      buffer[i] = stateArray[i] ^ stateArray[i + 5]
	    ^ stateArray[i + 10] ^ stateArray[i + 15]
        ^ stateArray[i + 20];

    for(i = 0; i < 5; i++) {
      temp = buffer[(i + 4) % 5] ^ ROTL(buffer[(i + 1) % 5], 1);
      for(j = 0; j < 25; j += 5)
        stateArray[j + i] ^= t;
    }

    /* Rho Pi */
    temp = stateArray[1];
    for(i = 0; i < 24; i++) {
      j = keccakf_pi_indices[i];
      buffer[0] = stateArray[j];
      stateArray[j] = ROTL(t, keccak_rho_distances[i]);
      temp = buffer[0];
    }

    /* Chi */
    for(j = 0; j < 25; j += 5) {
      for(i = 0; i < 5; i++)
        buffer[i] = stateArray[j + i];
        for(i = 0; i < 5; i++)
          stateArray[j + i] ^= (~buffer[(i + 1) % 5])
                               & buffer[(i + 2) % 5];
    }

    /* Iota */
    stateArray[0] ^= keccak_round_constants[round];
  }
}
\end{lstlisting}
\centering
\caption{Implementierung der KECCAK-p-Funktion}
\small Mit dieser Implementierung wurden die Tests in Kapitel \ref{cha:ergebnisse} durchgeführt.
\label{fig:keccak_impl}
\end{figure}




\chapter{Template Stuff that is still here}

\section{Some Template Comments}
\label{sec:comments}

\begin{outline}
  \1 It is recommended to use one sentence per line of the latex source code.
  That is a good compromise between (i) `diffs' when using repositories, and (ii) forward-/backward search between latex source code and pdf output.
  \1 Note that you can have multiple refs in the same \textbackslash cref block (e.g., \cref{sec:motivation,sec:comments,sec:statement,sec:results,fig:popcount}), but there must not be spaces after the commas.
  \1 Note that you should use \textbackslash Cref (upper-case C) at the beginning of a sentence and \textbackslash cref (lower-case c) in the middle of a sentence.
  They are defined differently, such that the upper-case C version does not use abbreviations (which is recommended for the beginning of a sentence), e.g., \cref{eq:node} vs. \Cref{eq:node}.
  \1 You can use the outline environment to collect itemized points before actually writing your text.
    \2 It helps structuring your ideas by simplifying indentation of items
    \3 Like here.
\end{outline}


\section{Problem Statement}
\label{sec:statement}

Based on a partitioning $P \subset 2^V$, i.e., $p_i, p_j \in P, p_i \ne p_j \Rightarrow p_i \cap p_j = \emptyset, \bigcup\limits_P = V$, an equivalence relation $\sim_P$ as well as the partition graph $G_P$ are defined as follows:
\begin{align}
  \sim_P &= \{(v_1,v_2) \in V | \exists p \in P: v_1\in p \wedge v_2 \in p\}\\
  G_P &= (V_P, E_P) = (V/\sim_P, \{ ([v_1]_{\sim_P}, [v_2]_{\sim_P}) |\: (v_1,v_2) \in E \})
  \label{eq:node}
\end{align}



\section{Results}
\label{sec:results}

Following is the discussion of obtained results.

\begin{figure}[ht]
  \center
  \includegraphics[scale=0.33]{searchtree.pdf}
  \caption{Topologically sorted DFG along with the complete search tree of the partition enumeration algorithm.}
  \label{fig:searchtree}
\end{figure}



\begin{table}
  \centering
  \begin{tabular}{lrrrr}
                &   types  &      types  &   atoms  &      atoms  \\
    SI          &  manual  &  generated  &  manual  &  generated  \\
    \hline
    htfour      &       1  &          4  &       8  &         81  \\
    satdfour    &       3  &          8  &      16  &        104  \\
    dctfour     &       2  &          9  &      12  &         90  \\
    sadsixteen  &       1  &          4  &      64  &        255  \\
  \end{tabular}
  \caption{Comparison of generated SI graphs vs. hand-crafted ones.}
  \label{tab:manualeval}
\end{table}


\begin{figure}
\lstset{language=C}
\begin{lstlisting}[label={lst:popcount}]
uint32_t popcount_a(uint32_t x)
{
  x -= ((x >> 1) & 0x55555555);
  x = (x & 0x33333333) + ((x >> 2) & 0x33333333);
  x = (x + (x >> 4)) & 0x0f0f0f0f;
  x += x >> 8;
  x += x >> 16;
  return x & 0x3f;
}
\end{lstlisting}
\caption{C-Code from \cite{warren2003hacker} to compute the number of set bits of a 32-bit value.}
\label{fig:popcount}
\end{figure}

